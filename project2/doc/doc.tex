\documentclass[a4paper, 11pt, fleqn]{scrartcl}

\usepackage[czech]{babel}
\usepackage[utf8]{inputenc}
\usepackage[T1]{fontenc}
\usepackage{times}
\usepackage[left=2cm, top=3cm, text={17cm, 24cm}]{geometry}
\usepackage[unicode, colorlinks, hypertexnames=false, citecolor=red]{hyperref}
\usepackage{fancyhdr}
\usepackage{lastpage}
\usepackage[shortlabels]{enumitem}
\usepackage{amssymb}
\usepackage{amsmath}
\usepackage{amsthm}
\usepackage{newtxtext, newtxmath}
\usepackage{graphicx}
\usepackage{tikz-uml}
\usepackage{changepage}

\usepackage[ruled,linesnumbered]{algorithm2e}

%%%%%%%%%%%%%%%%%%%%%%%%%%%%%%%%%%%%%%%%%%%%%%%%%%%%%%%%%%%%%%%%%%%%%
% Makra %%%%%%%%%%%%%%%%%%%%%%%%%%%%%%%%%%%%%%%%%%%%%%%%%%%%%%%%%%%%%

\newcommand{\KURZ}{Paralelní a distribuované algoritmy}
\newcommand{\AUTOR}{Bc. Michal Šedý (xsedym02)}
\newcommand{\NAZEV}{Implementace paralelního algoritmu K-means}


%%%%%%%%%%%%%%%%%%%%%%%%%%%%%%%%%%%%%%%%%%%%%%%%%%%%%%%%%%%%%%%%%%%%%
% Hlavička %%%%%%%%%%%%%%%%%%%%%%%%%%%%%%%%%%%%%%%%%%%%%%%%%%%%%%%%%%

\pagestyle{fancy}
\fancyhead[L]{\AUTOR}
\fancyhead[C]{\KURZ}
\fancyhead[R]{\today}

\fancyfoot[C]{}
\fancyfoot[R]{\thepage\,/\,\pageref*{LastPage}}

\setlength{\parindent}{0pt}
\setlength{\mathindent}{0pt}

%%%%%%%%%%%%%%%%%%%%%%%%%%%%%%%%%%%%%%%%%%%%%%%%%%%%%%%%%%%%%%%%%%%%%
% Text %%%%%%%%%%%%%%%%%%%%%%%%%%%%%%%%%%%%%%%%%%%%%%%%%%%%%%%%%%%%%%

\begin{document}

  \begin{center}
    {\Large \NAZEV}
  \end{center}

  Cílem druhého projektu do předmětu paralelní a distribuované algoritmy (PRL) bylo implementovat paralelní algoritmus K-means s využitím knihovny OpenMPI. Tato dokumentace popisuje využitý shlukovací algoritmu spolu s jeho složitosti, komunikaci mezi jednotlivými procesy a příklad použití vytvořeného programu.

  \section{K-means}
    V této sekci ukážeme sekvenční a paralelní algoritmus K-means a jejich složitosti. Shlukování bylo prováděno na posloupnosti jednorozměrných hodnot. Za účelem projektu jsou vytvářeny 4 shluky (v důkazech složitosti bude počítáno s $k = 4$). Jako první hodnoty středů shluků jsou zvoleny první 4 hodnoty shlukované posloupnosti.

    \subsection{Sekvenční K-means}

      Na vstupu sekvenčního algoritmu je posloupnost (množina) jednorozměrných bodů $points$ a počet tvořených shluků. Jako středy ($means$) shluků je vybráno $k$ prvních bodů z posloupnosti $points$. Každý shluk ($cluster$) obsahuje množinu přiřazených bodů. Pro $i \in \mathbb{N} > 1$ platí $\forall n, m \in \mathbb{N} \leq k: cluster_i(n) \cap cluster_i(m) \neq \emptyset \iff n = m$ a také $\bigcup_{\forall n \in \mathbb{N} \leq k} cluster_i(n) = points$. V každé iteraci algoritmu jsou body přiřazeny do shluku, který má nejbližší střed. Pokud během iterace nedošlo ke změně žádného shluku, je algoritmus ukončen.

      \begin{algorithm}[!h]
        \label{algo:slow}
        \SetAlgoLined
        \DontPrintSemicolon
        \caption{Sekvenční K-means}
        \KwIn{$points$, $k \in \mathbb{N}$}
        \KwOut{$cluster: k \rightarrow 2^{points}$}

        \vspace*{2mm}

        \ForAll{$p_n \in \{p_1, \dots, p_k\} \subseteq points$}
        {
          $mean(n) \leftarrow p_n$\;
          $cluster_0(n) \leftarrow \emptyset$\;
        }

        \vspace*{2mm}

        $i \leftarrow 1$\;
        \Repeat{$\forall n \in \mathbb{N} \leq k : cluster(n)_i = cluster_{i-1}(n)$}
        {
          \ForAll{$p \in points$}
          {
            $c \leftarrow arg min_{n \in \mathbb{N} : n \leq k}(|mean(n) - p|)$\;
            $cluster_i(c) \leftarrow cluster_i(c) \cup \{p\}$\;
          }

          \vspace*{2mm}

          \ForAll{$n \in \mathbb{N} : n \leq k$}
            {
              $means(n) \leftarrow \frac{1}{|cluster_i(n)|} \sum_{p \in cluster_i(n)}p$
            }
          $i \leftarrow i + 1$\;
        }

        \vspace*{2mm}

        \Return{$cluster_i$}
      \end{algorithm}

      \subsubsection{Časová složitost}
        Časová složitost sekvenčního algoritmu K-means, pro $K = 4$, je $\mathcal{O}(n^2)$, kde $n$ je počet shlukovaných bodů.

        \begin{proof}
          Časová složitost inicializace shluků (1--4) je $\mathcal{O}(k)$, kde $k$ je počet shluků. Složitost smyčky aktualizující shluky (6--15) je $\mathcal{O}(n + nk) = \mathcal{O}(n)$, kde $n$ je počet bodů. V \cite{slow} bylo ukázáno, že konvergence algoritmu nastává po $\mathcal{O}(n^{kd})$, kde $d$ je počet dimenzí. Tato hranice byla v \cite{fast} upřesněna pro speciální případy, kdy $k < 5$ a $d = 1$ na $\mathcal{O}(n)$ iterací. Z předchozího vyplývá, že celková časová složitost algoritmu \ref{algo:slow} je $\mathcal{O}(n^2)$.
        \end{proof}

      \subsubsection{Prostorová složitost}
        Prostorová složitost sekvenčního algoritmu K-means, pro $K = 4$, je $\mathcal{O}(n)$, kde $n$ je počet shlukovaných bodů.

        \begin{proof}
          Algoritmus využívá proměnnou $means$ s prostorovou složitostí $\mathcal{O}(k) = \mathcal{O}(1)$. Proměnné $cluster_i$ se dají znovu využívat, je potřeba uchovávat pouze $cluster_i$ a $cluster_{i-1}$. Jejich prostorová složitost je $\mathcal{O}(n)$. Celková prostorová složitost algoritmu je $\mathcal{O}(n)$.
        \end{proof}

    \subsection{Paralelní K-means}

      \begin{algorithm}
        \label{algo:fast}
        \SetAlgoLined
        \DontPrintSemicolon
        \caption{Paralelní K-means}
        \KwIn{$points$, $k \in \mathbb{N}$}
        \KwOut{$cluster: k \rightarrow 2^{points}$}

        \vspace*{2mm}

        \ForAll{$p_n \in \{p_1, \dots, p_k\} \subseteq points$}
        {
          $mean(n) \leftarrow p_n$\;
          $Broadcast(means(n))$\;
        }

        \vspace*{2mm}

        \ForAll(\textbf{in parallel}){$p_i \in \{p_1, \dots, p_n\} = points$}
        {
          $C^{(i)} \leftarrow 0$ \tcp*{Přiřaď $0$ do proměnné shluku $C$ procesu $i$.}
          $changed^{(i)} \leftarrow \textbf{True}$ \tcp*{Přiřaď \textbf{$True$} do proměnné $changed$ procesu $i$.}
        }

        \vspace*{2mm}

        $changed \leftarrow \textbf{True}$\;
        \While{$changed$}
        {
          \ForAll(\textbf{in parallel}){$p_i \in \{p_1, \dots, p_n\} = points$}
          {
            $c \leftarrow arg min_{n \in \mathbb{N} : n \leq k}(|mean(n) - p_i|)$\;
            $changed^{(i)} \leftarrow c \neq C^{(i)}$ \tcp*{Logický výraz}
          }

          $cluster\_size \leftarrow GetSizesOfClusters()$ \tcp*{Redukcí získej velikosti všech shluků.}
          $cluster\_sum \leftarrow GetSumInClusters()$ \tcp*{Redukcí získej sumu v každém shluku.}
          \ForAll{$n \in \mathbb{N} : n \leq k$}
          {
            $means(n) \leftarrow cluster\_sum(n) / cluster\_size(n)$\;
            $Broadcast(means(n))$\;
          }

          $changed \leftarrow ORReuce(changed^{(1)}, \dots, changed^{(n)})$ \tcp*{Redukce procesových $changed$.}

        }

        \vspace*{2mm}

        \ForAll{$n \in \mathbb{N} : n \leq k$}
        {
          $cluster(n) = \{p_i \in points\,|\, C^{(i)} = n\}$\;
        }

        \vspace*{2mm}

        \Return{$cluster$}

      \end{algorithm}

      Paralelní algoritmu K-means využívá $n$ procesů, kde $n$ je shodné s počtem shlukovaných bodů. Na vstupu algoritmu je posloupnost (množina) jednorozměrných bodů $points$ a počet tvořených shluků $k$. Jako středy ($means$) shluků je vybráno $k$ prvních bodů z posloupnosti $points$. Tyto středy jsou následně pomocí operace \textit{Broadcast} rozeslány ostatním procesům. Každý procesor obsluhuje jeden vstupní bod. Procesor $i$ obsluhuje bod $p_i$. Každý proces obsahuje lokální proměnné $C^{(i)}$ a $changed^{(i)}$. $C^{(i)}$ určuje příslušnost bodu $p_i$ do shluku. $changed^{(i)}$ určuje, zda v současné iteraci algoritmu změnil bod $p_i$ shluk. Po ukončení rozmístění bodů do shluků jsou pomocí operace \textit{Reduce} získány velikosti (počet bodů ve shluku) a sumy bodů v každém shluku. Pomocí těchto hodnot proces \textit{root} přepočítá hodnoty středů shluků a operací \textit{Broadcast} je rozešle všem procesům pro další iteraci výpočtu. Všechny lokální hodnoty $changed^{(i)}$ jsou operací \textit{ORReduce} zredukovány do proměnné $changed$. Pokud je $changed$ rovno \textit{True} (nějaký bod změnil shluk) je započata nová iterace. Výpočet  je ukončen pokud už žádným bod nemění shluk.

      \subsubsection{Časová složitost}
        Časová složitost paralelního algoritmu K-means, pro $K = 4$, je $\mathcal{O}(n\cdot log\, n)$, kde $n$ je počet shlukovaných bodů.

        \begin{proof}
          Časová složitost výběru středů a jejich rozeslání procesorům (1--4) je $\mathcal{O}(log\, n)$. Složitost inicializace procesových proměnných (5--8) je $\mathcal{O}(1)$. Smyčka pro výpočet nových středů má časovou složitost $\mathcal{O}(k) = \mathcal{O}(1)$. Časová složitost Reduce pro získání $cluster\_size$ (15), $cluster\_size$ (16) a $changed$ (21) je $\mathcal{O}(log\, n)$. Přepočítání středů shluků a jejich rozeslání procesům (17--20) má časovou složitost $\mathcal{O}(log\, n)$. Protože z \cite{fast} víme, že ke konvergenci pro jednorozměrná data a $k < 5$ dochází po $\mathcal{O}(n)$ iteracích hlavního cyklu (10--22), pak je celková časová složitost paralelního algoritmu $\mathcal{O}(n\cdot log\, n)$.
        \end{proof}


      \subsubsection{Cena}
        Cena paralelního algoritmu K-means, pro $K = 4$, je $\mathcal{O}(n^2\cdot log\, n)$, kde $n$ je počet shlukovaných bodů. Algoritmus není optimální.

        \begin{proof}
          Dříve jsme ukázali, že časová složitost paralelního algoritmu K-means je $\mathcal{O}(n\cdot log\, n)$, kde $n$ je počet shlukovaných bodů. Při použití $p$ procesů, kde $p = n$ je cena paralelního algoritmu $\mathcal{O}(pn\cdot log\, n) = \mathcal{O}(n^2\cdot log\, n)$. Lze viděl, že paralelní algoritmus není efektivní, protože cena sekvenčního algoritmu je $\mathcal{O}(n^2)$.
        \end{proof}

      \subsubsection{Prostorová složitost}
        Prostorová složitost paralelního algoritmu K-means, pro $K = 4$, je $\mathcal{O}(n)$, kde $n$ je počet shlukovaných bodů.

        \begin{proof}
          Prostorová složitost proměnné $means$ je $\mathcal{O}(k) = \mathcal{O}(1)$. Proměnná $cluster$, a všechny lokální proměnné $c$ a $changed$ mají dohromady prostorovou složitost $\mathcal{O}(3n) = \mathcal{O}(n)$. Proměnné $cluster\_size$ a $cluster\_sum$ mají dohromady prostorovou složitost $\mathcal{O}(1)$. Celková prostorová složitost paralelního algoritmu je tedy $\mathcal{O}(n)$.
        \end{proof}

  \section{Komunikace}

    Následující sekvenční diagram popisuje komunikaci procesů paralelního algoritmu K-means implementovaného v \texttt{parkmeans.cc}. Během výpočtu je využito $n$ procesů. Kořenový proces $p0$ příjme požadavek na shlukování bodů $points$. Inicializuje středy shluků a ty pomocí \textit{Broadcast} rozešle ostatním $n-1$ procesům.  Dále je každému procesům rozeslána hodnota $changed=True$. Každému procesu je pomocí \textit{Scatter} přidělen jeden bod. Smyčka (\textit{while}) je opakována, dokud nedojde ke konvergenci ($changed=False$). Ve smyčce kořenový proces $p0$ získá pomocí \textit{Reduce} počet a součet bodů v jednotlivých shlucích. K zjištění, zda některý bod změnil shluk využije kořenový proces \textit{ORReduceAll} a výslednou hodnotu $changed$ rozešle zpět všem procesům. Aktualizované hodnoty jsou pomocí \textit{Broadcast} opěr rozeslány ostatním $n-1$ procesům. Po ukončení smyčky získá kořenový proces rozdělení bodů do shluků operací \textit{Gather}.

    \newpage

    \begin{figure}[!ht]
      \centering
      \begin{tikzpicture}
        \begin{umlseqdiag}
          \umlobject[class=App]{app}
          \umlobject[class=Proc]{p0}
          \umlmulti[class=Proc]{p1}
          \begin{umlcall}[op={parkmeans(points)}, return=cluster]{app}{p0}
            \begin{umlcall}[op={Broadcast(mean)}]{p0}{p1}
            \end{umlcall}
            \begin{umlcall}[op={Scatter(points)}]{p0}{p1}
            \end{umlcall}
            \begin{umlcall}[op={Broadcast(\textnormal{changed=True})}]{p0}{p1}
            \end{umlcall}
            \begin{umlfragment}[type=while, label=\textnormal{changed=True}, name=while, inner xsep=12]
              \begin{umlcall}[op={CNTReduce(cluster)}]{p1}{p0}
              \end{umlcall}
              \begin{umlcall}[op={SUMReduce(cluster)}]{p1}{p0}
              \end{umlcall}
              \begin{umlcall}[op={Broadcast(mean)}]{p0}{p1}
              \end{umlcall}
              \begin{umlcall}[op={ORReduceAll(changed)}, return=changed]{p1}{p0}
              \end{umlcall}
            \end{umlfragment}
            \begin{umlcall}[op={Gather(cluster)}]{p1}{p0}
            \end{umlcall}
          \end{umlcall}
        \end{umlseqdiag}
      \end{tikzpicture}
      \caption{Sekvenční graf popisující komunikaci procesu v paralelním programu \texttt{parkmeans.cc}.}
    \end{figure}



  \section{Obsluha programu}
    Program načítá sekvenci nezáporných čísel (minimálně 4 a maximálně 32) o velikosti 1 byte ze souboru \texttt{numbers}. Tento soubor s $n$ čísly leze vytvořit příkazem \texttt{dd if=/dev/random bs=1 count=n of=numbers}. Program je nutné spouštět se stejným počtem procesů, jako je počet čísel určených ke shlukování. Středy shluků jsou inicializovány prvními 4 unikátními hodnotami. Pokud takové hodnoty neexistují, vytvoří se duplicitní středy shluku s tím, že hodnoty bodů budou přiřazovány pouze jednomu z nich.

    \subsection*{Příklad}
      Vstup: \texttt{1,10,50,1}\\
      Výstup:
      \begin{adjustwidth}{1em}{}
        \texttt{[1.0] 1}\\
        \texttt{[10.0] 10}\\
        \texttt{[50.0] 50}\\
        \texttt{[1.0]}
      \end{adjustwidth}

  \section{Závěr}
    Cílem tohoto projektu bylo implementovat s využitím OpenMPI paralelní shlukovací algoritmus K-means pro jednorozměrná data vyžívající 4 shluky ($K = 4$). Bylo ukázáno, že prostorová složitost sekvenčního i paralelního algoritmu je $\mathcal{O}(n)$, ked $n$ je počet shlukovaných bodů. Na základě \cite{fast} lze určit, že pro počet shluků menší 5 a~jednorozměrná data je časová složitost sekvenčního algoritmu $\mathcal{O}(n^2)$. Časová složitost paralelního algoritmu je $\mathcal{O}(n\cdot log\, n)$. V případě využití $n$ procesů je cena paralelního algoritmu $\mathcal{O}(n^2\cdot log n)$. Z těchto poznatků vyplývá, že paralelní algoritmus není optimální.

  \bibliographystyle{plain} % We choose the "plain" reference style
  \bibliography{doc} % Entries are in the refs.bib file

\end{document}


