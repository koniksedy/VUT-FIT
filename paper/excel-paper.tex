\documentclass{ExcelAtFIT}
%\documentclass[czech]{ExcelAtFIT} % when writing in CZECH
%\documentclass[slovak]{ExcelAtFIT} % when writing in SLOVAK

% \setlength{\fboxsep}{1.5pt}

% Command for color circle around text.
\usepackage{tikz}
\newcommand{\circledtext}[2][red]{%
    \tikz[baseline=(char.base)]{
        \node[shape=circle, draw, fill=#1, inner sep=0pt] (char) {\hspace*{0.15mm}\textbf{\textcolor{#1}{#2}}};}\hspace*{-1mm}
}

%--------------------------------------------------------
%--------------------------------------------------------
%	REVIEW vs. FINAL VERSION
%--------------------------------------------------------

%   LEAVE this line commented out for the REVIEW VERSIONS
%   UNCOMMENT this line to get the FINAL VERSION
\ExcelFinalCopy


%--------------------------------------------------------
%--------------------------------------------------------
%	PDF CUSTOMIZATION
%--------------------------------------------------------

\hypersetup{
	pdftitle={Repetitive Substructures for Efficient Representation of Automata},
	pdfauthor={Michal Šedý},
	pdfkeywords={Nondeterministic Finite Automata, Nondeterministic Pushdown Automata, Regular Expressions, Network Intrusion Detection Systems}
}

\lstset{
	backgroundcolor=\color{white},   % choose the background color; you must add \usepackage{color} or \usepackage{xcolor}; should come as last argument
	basicstyle=\footnotesize\tt,        % the size of the fonts that are used for the code
}

%--------------------------------------------------------
%--------------------------------------------------------
%	ARTICLE INFORMATION
%--------------------------------------------------------

\ExcelYear{2024}

\PaperTitle{Repetitive Substructures for \\Efficient Representation of Automata}

\Authors{Michal Šedý*}
\affiliation{*%
  \href{mailto:xsedym02@stud.fit.vutbr.cz}{xsedym02@stud.fit.vutbr.cz},
  \textit{Faculty of Information Technology, Brno University of Technology}}
%%%%--------------------------------------------------------
%%%% in case there are multiple authors, use the following fragment instead
%%%%--------------------------------------------------------
%\Authors{Jindřich Novák*, Janča Dvořáková**}
%\affiliation{*%
%  \href{mailto:xnovak00@stud.fit.vutbr.cz}{xnovak00@stud.fit.vutbr.cz},
%  \textit{Faculty of Information Technology, Brno University of Technology}}
%\affiliation{**%
%  \href{mailto:xdvora00@stud.fit.vutbr.cz}{xdvora00@stud.fit.vutbr.cz},
%  \textit{Faculty of Information Technology, Brno University of Technology}}


%--------------------------------------------------------
%--------------------------------------------------------
%	ABSTRACT and TEASER
%--------------------------------------------------------

\Abstract{
	Nondeterministic finite automata are widely used across almost every field of computer science, such as for the representation of regular expressions, in network intrusion detection systems for monitoring high-speed networks, in abstract regular model checking, program verification, or even in bioinformatics for searching sequences of nucleotides in DNA. To obtain smaller automata, and thus reduce computational resources, state-of-the-art minimization techniques, such as state merging and transition pruning, are used. However, these methods can still leave duplicate substructures in the resulting automata. This work presents a~novel approach to automata minimization based on the transformation of an NFA into a~nondeterministic pushdown automaton. The transformation identifies and represents similar substructures only once by so-called procedures. By doing so, we can further reduce automata by up to 60\%.
}

%--------------------------------------------------------
%--------------------------------------------------------
%--------------------------------------------------------
%--------------------------------------------------------
\begin{document}

\startdocument

%--------------------------------------------------------
%--------------------------------------------------------
%	ARTICLE CONTENTS
%--------------------------------------------------------

%%%%%%%%%%%%%%%%%%%%%%%%%%%%%%%%%%%%%%%%%%%%%%%%%%%%%%%%%%%%
%%%%%%%%%%%%%%%%%%%%%%%%%%%%%%%%%%%%%%%%%%%%%%%%%%%%%%%%%%%%
%%%%%%%%%%%%%%%%%%%%%%%%%%%%%%%%%%%%%%%%%%%%%%%%%%%%%%%%%%%%
%%%%%%%%%%%%%%%%%%%%%%%%%%%%%%%%%%%%%%%%%%%%%%%%%%%%%%%%%%%%
\section{Introduction}
	Nondeterministic finite automata (NFAs), as their name suggests, can nondeterministically transition from one state to multiple states based on the same input. This property allows NFAs to represent languages more compactly than their deterministic counterparts, which can only be in one state at a time. Despite their hard minimization, NFAs find applications in many fields of computer science, such as representing regular expressions, network intrusion detection systems for monitoring high-speed networks \cite{FPGA_based_network_scaning, ApproxRed}, abstract regular model checking \cite{ARMC}, verifying programs that manipulate strings \cite{String_constraints_for_ver}, or decision procedures in the WS1S and WS2S logic \cite{On_equivalence_checking, Nested_antichains_for_WS1S}.

	\subsection*{Minimization Techniques}
		To reduce computational resources when working with NFAs, it is crucial to reduce their size. For this purpose, the state merging \cite{Oldest_Merge,Simulation_based_minimization,On_nfa_reduction} and transition pruning \cite{Simulation_based_minimization, Lorenzo_prunning_saturation} techniques are being used. The state merging technique can merge two states if one of them fully covers the logic of the other. On the other hand, transition pruning removes a transition if there already exists a duplicate transition with the same logic. These methods are implemented in the state-of-the-art tool RABIT/Reduce \cite{RABIT}.

	\subsection*{Repetitive Substructures}
		Despite the good reduction potential that standard minimization techniques offer, the resulting automata can still contain redundant substructures. These automata often represent regular expressions, such as those used in network intrusion detection systems (NIDSs) for network traffic scanning. They are constructed as the union of regular expressions. Additionally, there are types of automata that cannot be minimized by these standard methods at all.

	\subsection*{Our Novel Approach}
		In our work, we present a novel reduction approach that involves transforming a NFA into a nondeterministic pushdown automaton (NPDA) that utilizes a~stack. This approach identifies similar substructures within the automaton and represents them only once using so-called procedures. The stack is then utilized to track the states from which the procedure has been entered and where to return. This transformation can be likened to converting a purely sequential program into one that uses functions and a call stack. By applying our approach to the results of standard minimization techniques, we were able to achieve an additional reduction of up to 60\% in both the states and the transitions of the automaton.

%%%%%%%%%%%%%%%%%%%%%%%%%%%%%%%%%%%%%%%%%%%%%%%%%%%%%%%%%%%%
%%%%%%%%%%%%%%%%%%%%%%%%%%%%%%%%%%%%%%%%%%%%%%%%%%%%%%%%%%%%
%%%%%%%%%%%%%%%%%%%%%%%%%%%%%%%%%%%%%%%%%%%%%%%%%%%%%%%%%%%%
%%%%%%%%%%%%%%%%%%%%%%%%%%%%%%%%%%%%%%%%%%%%%%%%%%%%%%%%%%%%
\section{Motivation}
	The automaton representing a regular expression \texttt{(.*new XMLHttpRequest.*file://)|(.*file://\\.*new XMLHttpRequest)} from network intrusion detection system Snort \cite{Snort} is shown in \fbox{Figure 1}. Besides the epsilon transitions and \texttt{.*}, this is the most minimal form that can be achieved by standard minimization techniques. This is caused by the lack of language inclusions as  \textit{Request} and \textit{File} substructures are compactly different. As a result, the automaton contains two substructures, each of which has redundant copy, making the NFA representation unefficient.

%%%%%%%%%%%%%%%%%%%%%%%%%%%%%%%%%%%%%%%%%%%%%%%%%%%%%%%%%%%%
%%%%%%%%%%%%%%%%%%%%%%%%%%%%%%%%%%%%%%%%%%%%%%%%%%%%%%%%%%%%
%%%%%%%%%%%%%%%%%%%%%%%%%%%%%%%%%%%%%%%%%%%%%%%%%%%%%%%%%%%%
%%%%%%%%%%%%%%%%%%%%%%%%%%%%%%%%%%%%%%%%%%%%%%%%%%%%%%%%%%%%
\section{One Procedure No Duplicate}
	At this point, we identified repetitive substructures for \textit{Request} and \textit{File}, which represent duplicate information. Each of these substructures will be represented by a corresponding procedure. This transformation into procedures can be seen in \fbox{Figure 2}.

	\subsection*{Entering the Procedure}
		To recognize if the procedure for \textit{Request} has been entered from state 1 as the first part of the regular expression or from state 6 as the second part, the symbol \circledtext[orange]{1} or \circledtext[black!20!green]{6} is pushed onto the stack, respectively.

	\subsection*{Changing Procedures}
		When directly transferring from the \textit{Request} procedure into the \textit{File} procedure, it is necessary to ensure that this transition will be used only once and only after first entering the \textit{Request} from state 1. This is done by testing for the symbol \circledtext[orange]{1} at the top of the stack. If the top matches the required symbol, it is replaced with the symbol \circledtext[white!20!red]{5}. The same approach applies for the transition from the \textit{File} procedure into the \textit{Request} procedure.

	\subsection*{Returning From the Procedure}
		Transitions exiting the \textit{Request} procedure can only be used when the stack contains the corresponding symbol that is popped afterward. For the transition between states 8 and 9, it is the symbol \circledtext[black!20!green]{6}, indicating that the \textit{Request} procedure represents the last part of the regular expression that started with the \textit{File}.


%%%%%%%%%%%%%%%%%%%%%%%%%%%%%%%%%%%%%%%%%%%%%%%%%%%%%%%%%%%%
%%%%%%%%%%%%%%%%%%%%%%%%%%%%%%%%%%%%%%%%%%%%%%%%%%%%%%%%%%%%
%%%%%%%%%%%%%%%%%%%%%%%%%%%%%%%%%%%%%%%%%%%%%%%%%%%%%%%%%%%%
%%%%%%%%%%%%%%%%%%%%%%%%%%%%%%%%%%%%%%%%%%%%%%%%%%%%%%%%%%%%
\section{Experiments}
	We tested our reduction method on parametric and real-world regular expressions used in network filtering. The highest reductions were obviously achieved on larger automata, as there is a greater likelihood of similar substructures existing. Since our tool is designed to follow after standard reduction methods, the percentage reduction is calculated relative to the resulting automata of RABIT/Reduce.

	\subsection{Parametric Regular Expressions}
		The set of 3'656 automata, with an average of 207 states and 2'584 transitions, was obtained from four families of parametric regular expressions \cite{Regex_param}. The reduction ratio of states and transitions can be seen in \fbox{Figure 3}. It can be observed that, on average, our tool achieved a reduction of 52.5\% in the number of states. The x-axis of the graph represents the size of RABIT/Reduce results (the input of our tool), while the y-axis represents achieved reduction ratio. The graph is enhanced with temperature coloring and a~distribution function for each axis.

	\subsection{Network Intrusion Detection Systems}
		To test our reduction algorithm on real-world examples, seven automata were created as unions of sets of regular expressions from seven different families of Snort rules. The results of reduction performed by standalone usage of the RABIT/Reduce tool ($RAB$) and with the additional application of our method based on procedures ($\mathit{Proc}$) are shown in \fbox{Table 1}. The two most significant results are highlighted. The best obtained result achieved a reduction of 43.5\% in states and 60.3\% in transitions.

%%%%%%%%%%%%%%%%%%%%%%%%%%%%%%%%%%%%%%%%%%%%%%%%%%%%%%%%%%%%
%%%%%%%%%%%%%%%%%%%%%%%%%%%%%%%%%%%%%%%%%%%%%%%%%%%%%%%%%%%%
%%%%%%%%%%%%%%%%%%%%%%%%%%%%%%%%%%%%%%%%%%%%%%%%%%%%%%%%%%%%
%%%%%%%%%%%%%%%%%%%%%%%%%%%%%%%%%%%%%%%%%%%%%%%%%%%%%%%%%%%%
\section{Conclusion}
	In this work, we introduced a novel approach to automata reduction. This reduction transforms NFAs into NPDAs, noting the similarity with transforming a~pure sequential program into a program with functions and call stack. Applying our reduction approach to the results of the state-of-the-art reduction tool RABIT/Reduce resulted in reductions of up to 52.5\% in states and up to 60.3\% in transitions. These results suggest that our approach could significantly contribute to the reduction of automata in the future.



%--------------------------------------------------------
%--------------------------------------------------------
%--------------------------------------------------------
%	REFERENCE LIST
%--------------------------------------------------------
%--------------------------------------------------------
\phantomsection
\bibliographystyle{unsrt}
\bibliography{bibliography}

%--------------------------------------------------------
%--------------------------------------------------------
%--------------------------------------------------------
\end{document}