\documentclass{ExcelAtFIT}
%\documentclass[czech]{ExcelAtFIT} % when writing in CZECH
%\documentclass[slovak]{ExcelAtFIT} % when writing in SLOVAK

\setlength{\fboxsep}{1.5pt}

% Command for color circle around text.
\usepackage{tikz}
\newcommand{\circledtext}[2][red]{%
    \tikz[baseline=(char.base)]{
        \node[shape=circle, draw, fill=#1, inner sep=0pt] (char) {\hspace*{0.15mm}\textbf{\textcolor{#1}{#2}}};}\hspace*{-1mm}
}

%--------------------------------------------------------
%--------------------------------------------------------
%	REVIEW vs. FINAL VERSION
%--------------------------------------------------------

%   LEAVE this line commented out for the REVIEW VERSIONS
%   UNCOMMENT this line to get the FINAL VERSION
\ExcelFinalCopy


%--------------------------------------------------------
%--------------------------------------------------------
%	PDF CUSTOMIZATION
%--------------------------------------------------------

\hypersetup{
	pdftitle={Repetitive Substructures for Efficient Representation of Automata},
	pdfauthor={Michal Šedý},
	pdfkeywords={Nondeterministic Finite Automata, Nondeterministic Pushdown Automata, Regular Expressions, Network Intrusion Detection Systems}
}

\lstset{
	backgroundcolor=\color{white},   % choose the background color; you must add \usepackage{color} or \usepackage{xcolor}; should come as last argument
	basicstyle=\footnotesize\tt,        % the size of the fonts that are used for the code
}

%--------------------------------------------------------
%--------------------------------------------------------
%	ARTICLE INFORMATION
%--------------------------------------------------------

\ExcelYear{2024}

\PaperTitle{Repetitive Substructures for \\Efficient Representation of Automata}

\Authors{Michal Šedý*}
\affiliation{*%
  \href{mailto:xsedym02@stud.fit.vutbr.cz}{xsedym02@stud.fit.vutbr.cz},
  \textit{Faculty of Information Technology, Brno University of Technology}}
%%%%--------------------------------------------------------
%%%% in case there are multiple authors, use the following fragment instead
%%%%--------------------------------------------------------
%\Authors{Jindřich Novák*, Janča Dvořáková**}
%\affiliation{*%
%  \href{mailto:xnovak00@stud.fit.vutbr.cz}{xnovak00@stud.fit.vutbr.cz},
%  \textit{Faculty of Information Technology, Brno University of Technology}}
%\affiliation{**%
%  \href{mailto:xdvora00@stud.fit.vutbr.cz}{xdvora00@stud.fit.vutbr.cz},
%  \textit{Faculty of Information Technology, Brno University of Technology}}


%--------------------------------------------------------
%--------------------------------------------------------
%	ABSTRACT and TEASER
%--------------------------------------------------------

\Abstract{
	Nondeterministic finite automata are widely used across almost every field of computer science, such as for a representation of regular expressions, in network intruction detection systems for monitoring high-speed networks, in abstract regular model checking, program verification, or even in bioinformatics for searching sequences of nucleotides in DNA. To obtain smaller automata and thus reduce computational resources, the state-of-the-art minimization techniques, such as state merging and transition prunning, are used. However, these methods can still leave duplicate substructures in the resulting automata. This work presents a novelty approach to automata minimization based on the transformation of a NFA into a~nondeterministic pushdown automaton. The transformation identifies and represents similar substructures only once by so called procedures. By this, we can further reduce automata by up to another 50\%.
}



%--------------------------------------------------------
%--------------------------------------------------------
%--------------------------------------------------------
%--------------------------------------------------------
\begin{document}

\startdocument

%--------------------------------------------------------
%--------------------------------------------------------
%	ARTICLE CONTENTS
%--------------------------------------------------------

%%%%%%%%%%%%%%%%%%%%%%%%%%%%%%%%%%%%%%%%%%%%%%%%%%%%%%%%%%%%
%%%%%%%%%%%%%%%%%%%%%%%%%%%%%%%%%%%%%%%%%%%%%%%%%%%%%%%%%%%%
%%%%%%%%%%%%%%%%%%%%%%%%%%%%%%%%%%%%%%%%%%%%%%%%%%%%%%%%%%%%
%%%%%%%%%%%%%%%%%%%%%%%%%%%%%%%%%%%%%%%%%%%%%%%%%%%%%%%%%%%%
\section{Introduction}
	Nondeterministic finite automat (NFA), as its name suggests, can nondeterministically traverse from one state into multiple states based on the same input. This allows NFAs to represent the language more compactly than its deterministic counterpart that can be only at one state at the time. Despite the complicated minimization caused by nondeterminism, NFAs are applied in many fields of computer science, such as representing regular expressions, network intrusion detection systems for monitoring high-speed networks \cite{FPGA_based_network_scaning, ApproxRed}, in abstract regular model checking \cite{ARMC}, in verifying programs that manipulate strings \cite{String_constraints_for_ver} or in decision procedures in the WS1S and WS2S logic \cite{On_equivalence_checking, Nested_antichains_for_WS1S}.

	\subsection*{Minimization techniques}
		To reduce computational resources when working with NFAs, it is crucial to reduce their size. For this purpose, the state merging \cite{Oldest_Merge,Simulation_based_minimization,On_nfa_reduction} and transition pruning \cite{Simulation_based_minimization, Lorenzo_prunning_saturation} techniques are being used. The state merging technique can merge two states if one of them fully covers the logic of the other. On the other hand the transition pruning removes a transition if there already exists a duplicit transition with the same logic. This minimization approaches are implemented in state-of-the-art tool RABIT/Reduce \cite{RABIT}.

	\subsection*{There are Repetitive Substructures}
		Despite the good minimization potential that standard minimization techniques carry, the resulting automata can still contain redundant substructures. This automata ofter represents a regular expressions. For example an automaton for network traffic scanning containing the union of regular expression from network intrusion detection systems (NIDS). There are also types of automata that cannot be minimized by this methods at all.

	\subsection*{Our Novelty Approach}
		In our work we present a novelty reduction approach, that incorporates a transformation of NFA to a nondeterministic pushdown automaton (NPDA) that uses a stack. The method identifies automaton's substructures, with the similar logic, and represents them only once by so-called procedure. The stack is then used to denote from which state the procedure have been entered and where to return af the end. This transformation can be understood as the conversion of a purely sequential program to program that uses functions and call stack. By applying our approach on results of the standard minimization techniques we were able to achieve an additional reduction of automaton by up to 50\% states and transitions.

%%%%%%%%%%%%%%%%%%%%%%%%%%%%%%%%%%%%%%%%%%%%%%%%%%%%%%%%%%%%
%%%%%%%%%%%%%%%%%%%%%%%%%%%%%%%%%%%%%%%%%%%%%%%%%%%%%%%%%%%%
%%%%%%%%%%%%%%%%%%%%%%%%%%%%%%%%%%%%%%%%%%%%%%%%%%%%%%%%%%%%
%%%%%%%%%%%%%%%%%%%%%%%%%%%%%%%%%%%%%%%%%%%%%%%%%%%%%%%%%%%%
\section{Motivation}
	The automaton representing a regular expression \texttt{(.*new XMLHttpRequest.*file://)|(.*file://\\.*new XMLHttpRequest)} from network intrusion detection system Snort \cite{Snort} is shown in \fbox{Figure 1}. Besides the epsilon transitions and \texttt{.*}, this is the most minimal form that can be achieved by standard minimization techniques. This is caused by the lack of language inclusions as  \textit{Request} and \textit{File} substructures are compactly different. As a result the automaton contains two substructures where each has one redundant copy, making the NFA representation unefficient.

%%%%%%%%%%%%%%%%%%%%%%%%%%%%%%%%%%%%%%%%%%%%%%%%%%%%%%%%%%%%
%%%%%%%%%%%%%%%%%%%%%%%%%%%%%%%%%%%%%%%%%%%%%%%%%%%%%%%%%%%%
%%%%%%%%%%%%%%%%%%%%%%%%%%%%%%%%%%%%%%%%%%%%%%%%%%%%%%%%%%%%
%%%%%%%%%%%%%%%%%%%%%%%%%%%%%%%%%%%%%%%%%%%%%%%%%%%%%%%%%%%%
\section{One Procedure No Duplicate}
	At this point we identify repetitive substructure for \textit{Request} and \textit{File}, that represents a duplicit information. Each of this substructures is going to be represented by corresponding procedure. This transformation to procedures can be see in \fbox{Figure 2}.

	\subsection*{Entering the Procedure}
		To recognize if the procedure for \textit{Request} has been entered from state 1 as the first part of the expression or from state 6 as the second part, the symbol \circledtext[orange]{1} respective \circledtext[black!20!green]{6} is pushed onto the stack.

	\subsection*{Changing Procedures}
		When directly transferring from \textit{Request} procedure into \textit{File} it is necessary to ensure, that this transition will be used only once and only after first entering \textit{Request} from the state 1. This is done by testing for the symbol \circledtext[orange]{1} on the top of the stack. If the top matches required symbol, it is replaced with the symbol \circledtext[white!20!red]{5}. The same goes for the transition from procedure \textit{File} into \textit{Request}.

	\subsection*{Returning From the Procedure}
		Transitions exiting the procedure \textit{Request} can be used only whe the stack contains corresponding symbol that is popped after. For the transition between states 8 and 9, it is the symbol \circledtext[black!20!green]{6}, meaning that the substructure for \textit{Request} represents the last part of regular expression that started with \textit{File}.


%%%%%%%%%%%%%%%%%%%%%%%%%%%%%%%%%%%%%%%%%%%%%%%%%%%%%%%%%%%%
%%%%%%%%%%%%%%%%%%%%%%%%%%%%%%%%%%%%%%%%%%%%%%%%%%%%%%%%%%%%
%%%%%%%%%%%%%%%%%%%%%%%%%%%%%%%%%%%%%%%%%%%%%%%%%%%%%%%%%%%%
%%%%%%%%%%%%%%%%%%%%%%%%%%%%%%%%%%%%%%%%%%%%%%%%%%%%%%%%%%%%
\section{Experiments}
	We tested our reduction method on parametric and real-world regular expressions from network filtering. The higher reduction was achieved obviously on larger automata as there is a bigger change for existence of similar substructures. Because our tool is concepted as next level that follows after standard reduction methods, the percentage reduction is referred to the resulting automata of RABIT/Reduce.

	\subsection{Parametric Regular Expressions}
		The set of 3'656 automata with 207 states and 2'584 transitions on average was obtained from four different families of parametric regular expression \cite{Regex_param}. The reduction ratio of states and transitions can be seen in \fbox{Figure 3}. In can be seen, that out tool performed 50\% reduction on average. Graphs's x-axis represents the size of RABIT/Reduce results (input of out tool) and y-axis represents reduction against the its input. The graph is enhanced with temperature coloring with a distribution function for each axis.

	\subsection{Network Intrusion Detection Systems}
		To test out reduction algorithm on read-world examples, seven automata created as unions of set of regular expressions, from seven different families of Snort rules, have been selected. The results of reduction performed by standalone usage of RABIT/Reduce ($RAB$) tool and with additional application of our method based on procedures with the two most significant highlighted results are shown in \fbox{Table 1}. The best result of application of our approach to RABIT/Reduce results achieved reduction of 43.5\% of states and 60.3\% of transitions.

%%%%%%%%%%%%%%%%%%%%%%%%%%%%%%%%%%%%%%%%%%%%%%%%%%%%%%%%%%%%
%%%%%%%%%%%%%%%%%%%%%%%%%%%%%%%%%%%%%%%%%%%%%%%%%%%%%%%%%%%%
%%%%%%%%%%%%%%%%%%%%%%%%%%%%%%%%%%%%%%%%%%%%%%%%%%%%%%%%%%%%
%%%%%%%%%%%%%%%%%%%%%%%%%%%%%%%%%%%%%%%%%%%%%%%%%%%%%%%%%%%%
\section{Conclusion}
	In this work we introduced a novelty approach to automata reduction. The reduction transforms NFA into NPDA using a similarity with transforming sequential program into a program with functions. Applying our reduction approach as on results of state-of-the-art reduction tool RABIT/Reduce resulted in up to 52.5\% of states and up to 60.3\% of transitions. This results indicates that our approach can play, in the future, an important role in the automata reduction.



%--------------------------------------------------------
%--------------------------------------------------------
%--------------------------------------------------------
%	REFERENCE LIST
%--------------------------------------------------------
%--------------------------------------------------------
\phantomsection
\bibliographystyle{unsrt}
\bibliography{bibliography}

%--------------------------------------------------------
%--------------------------------------------------------
%--------------------------------------------------------
\end{document}