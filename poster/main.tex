\documentclass[25pt, a1paper, portrait]{tikzposter}
\title{\parbox{0.8\linewidth}{\centering Repetitive Substructures for \\Efficient Representation of Automata}}
% \author{\vspace*{0.5em}Michal Šedý\vspace*{-0.5em}}
\author{\vspace*{1em}Michal Šedý}
\date{\today}
\institute{\normalsize supervised by doc. Mgr. Lukáš Holík, Ph.D.}

% Packages
\usepackage{url}
\usepackage{subfigure}
\usepackage{blindtext}
\usepackage{comment}
\usepackage{colortbl}
\usepackage{ctable}
\usepackage{tikz}
\usetikzlibrary{arrows,shapes,automata,backgrounds,petri,positioning}
\usetikzlibrary{decorations.pathmorphing}
\usetikzlibrary{decorations.shapes}
\usetikzlibrary{decorations.text}
\usetikzlibrary{decorations.fractals}
\usetikzlibrary{decorations.footprints}
\usetikzlibrary{shadows}
\usetikzlibrary{shapes.symbols}
\usetikzlibrary{shapes.callouts}
\tikzposterlatexaffectionproofoff


\usepackage{etoolbox}
\renewcommand\refname{}
\patchcmd{\thebibliography}{\section*{\refname}}{}{}{}


% Command for color circle around text.
\newcommand{\circledtext}[2][red]{%
    \tikz[baseline=(char.base)]{
        \node[shape=circle,draw,fill=#1,inner sep=0pt] (char) {\hspace*{0.15mm}\textbf{\textcolor{#1}{#2}}};}\hspace*{-2mm}
}

% Counter and environment for tables
\newcounter{tablecounter}
\newenvironment{tikztable}[1][]{%
  \def \rememberparameter{#1}%
  \refstepcounter{tablecounter}%
  \vspace{0.75em}}
  {
    \begin{center}
        \ifx\rememberparameter\@empty
        \else %nothing
        {\small Tab.~\thetablecounter: \rememberparameter \par\medskip}
        \fi
    \end{center}
}

% Set poster template
\usetheme{Envelope}
\useblockstyle[titleleft]{TornOut}
\colorlet{blocktitlefgcolor}{black!20!colorOne}
\usenotestyle{Sticky}

%%%%%%%%%%%%%%%%%%%%%%%%%%%%%%%%%%%%%%%%%%%%%%%%%%%%%%%%%%%%%%%%%%%%%%%%%%%%%%%
%%%%%%%%%%%%%%%%%%%%%%%%%%%%%%%%%%%%%%%%%%%%%%%%%%%%%%%%%%%%%%%%%%%%%%%%%%%%%%%
%%%%%%%%%%%%%%%%%%%%%%%%%%%%%%%%%%%%%%%%%%%%%%%%%%%%%%%%%%%%%%%%%%%%%%%%%%%%%%%
% DOCUMENT
%%%%%%%%%%%%%%%%%%%%%%%%%%%%%%%%%%%%%%%%%%%%%%%%%%%%%%%%%%%%%%%%%%%%%%%%%%%%%%%
%%%%%%%%%%%%%%%%%%%%%%%%%%%%%%%%%%%%%%%%%%%%%%%%%%%%%%%%%%%%%%%%%%%%%%%%%%%%%%%
%%%%%%%%%%%%%%%%%%%%%%%%%%%%%%%%%%%%%%%%%%%%%%%%%%%%%%%%%%%%%%%%%%%%%%%%%%%%%%%
\begin{document}

\maketitle[width=2\textwidth]

% Add logos
\node[below right=1.5cm and 11cm] at (TP@title) {\includegraphics[width=0.28\textwidth]{images/excel-fit-2024-logo.pdf}};
\node[below left=1.5cm and 11cm] at (TP@title) {\includegraphics[width=0.28\textwidth]{images/FIT_color_CMYK_EN.pdf}};

%%%%%%%%%%%%%%%%%%%%%%%%%%%%%%%%%%%%%%%%%%%%
%%%%%%%%%%%%%%%%%%%%%%%%%%%%%%%%%%%%%%%%%%%%
% MOTIVATION + PROCEDURES
%%%%%%%%%%%%%%%%%%%%%%%%%%%%%%%%%%%%%%%%%%%%
%%%%%%%%%%%%%%%%%%%%%%%%%%%%%%%%%%%%%%%%%%%%
\begin{columns}
    \column{0.4}
    \block{Motivation} {
        In many automata, especially those representing regular expressions, there exist repetitive substructures that cannot be eliminated using the state-of-the-art tool RABIT/Reduce \cite{RABIT}. This automaton is depicted in Figure~\ref{fig:http} below.

        \begin{tikzfigure}[\textsc{Automaton with duplicate substructures.}]
            \centering
            \begin{tikzpicture}[every initial by arrow/.style={, line width=1.5pt}, shorten >=1pt, node distance=3cm, auto, on grid, initial text=,font=\small,>=stealth',
                every state/.style={minimum size=7mm, line width=1.5pt}]
                \node[state, initial above , initial text=start] (0) {$0$};
                \node[cloud, line width=1.5pt, aspect=4, draw, cloud puffs=20, cloud puff arc=90, below left=4cm and 5cm of 0] (1) {\hspace*{-1.5em}\texttt{.*new XMLHttpRequest}\hspace*{-1.5em}};
                \node[cloud, line width=1.5pt, aspect=4, draw, cloud puffs=10, cloud puff arc=90, below right=4cm and 5cm of 0] (2) {\texttt{.*file://}};
                \node[state, below left=2.6cm and 5cm of 0, fill=orange] (11) {$\textbf{1}$};
                \node[state, below right=2.6cm and 5cm of 0, fill=white!50!blue] (22) {$\textbf{2}$};
                \node[state, below=2.8cm of 11, fill=orange] (33) {$\textbf{3}$};
                \node[state, below=2.8cm of 22, fill=white!50!blue] (44) {$\textbf{4}$};
                \node[cloud, line width=1.5pt, aspect=4, draw, cloud puffs=10, cloud puff arc=90, below=4.2cm of 33] (5) {\texttt{.*file://}};
                \node[cloud, line width=1.5pt, aspect=4, draw, cloud puffs=20, cloud puff arc=90, below=4.2cm of 44] (6) {\hspace*{-1.5em}\texttt{.*new XMLHttpRequest}\hspace*{-1.5em}};
                \node[state, below=2.8cm of 33, fill=white!20!red] (55) {$\textbf{5}$};
                \node[state, below=2.8cm of 44, fill=black!20!green] (66) {$\textbf{6}$};
                \node[state, below=2.8cm of 55, fill=white!20!red] (77) {$\textbf{7}$};
                \node[state, below=2.8cm of 66, fill=black!20!green] (88) {$\textbf{8}$};
                \node[state, accepting, below right=2.6cm and 5cm of 77] (9) {$9$};

                \path[->, line width=1.5pt]
                          (0) edge[out=west, in=north] node[above] {$\varepsilon$} (11)
                          (0) edge[out=east, in=north] node[above] {$\varepsilon$} (22)
                          (33) edge node[left] {$\varepsilon$} (55)
                          (44) edge node[right] {$\varepsilon$} (66)
                          (77) edge[out=south, in=west] node[below] {$\varepsilon$} (9)
                          (88) edge[out=south, in=east] node[below] {$\varepsilon$} (9);
            \end{tikzpicture}
            \label{fig:http}
        \end{tikzfigure}

        We propose a novel approach based on pushdown automata and so-called procedures, which represent repetitive substructures only once.
    }

    \note[
        rotate=12,
        targetoffsetx=6cm,
        targetoffsety=-11cm,
        width=0.122\linewidth
        ]
        {\small\textit{Usage of a pushdown automaton and procedures is analogous to the call stack and functions from programming languages.}}

    \column{0.6}
    \block{One Procedure No Duplicates} {
        To represent automata efficiently, without duplicate substructures, we introduce a new concept called procedures. Each set of similar substructures is represented by one procedure. The automaton uses a stack to determine the state from which the procedure is entered and the state to which it should return. The symbol on the stack can also serve to guard transitions that are specific to certain substructures represented by the procedure.

        \begin{tikzfigure}[\textsc{Automaton with two procedures and no duplicate substructures.}]
            \centering
            \begin{tikzpicture}[every initial by arrow/.style={line width=1.5pt}, shorten >=1pt, node distance=3cm, auto, on grid, initial text=,font=\small,>=stealth',
                every state/.style={minimum size=7mm, line width=1.5pt}]
                \node[state, initial, initial text=start] (0) {$0$};
                \node[cloud, line width=1.5pt, aspect=6, draw, cloud puffs=30, cloud puff arc=90, above right=4cm and 9cm of 0] (1) {\texttt{.*new XMLHttpRequest}};
                \node[cloud, line width=1.5pt, aspect=4, draw, cloud puffs=30, cloud puff arc=90, below right=4cm and 9cm of 0] (2) {\texttt{.*file://}};
                \node[state, above right=4cm and 3.9cm of 0, fill=orange] (11) {$\textbf{1}$};
                \node[state, below right=4cm and 5.7cm of 0, fill=white!50!blue] (22) {$\textbf{2}$};
                \node[state, right=10.2cm of 11, fill=black!20!green] (33) {$\textbf{8}$};
                \node[state, right=6.6cm of 22, fill=white!20!red] (44) {$\textbf{7}$};
                \node[state, above right=1.2cm and 2cm of 22, fill=white!20!red] (66) {$\textbf{5}$};
                \node[state, above left=1.2cm and 2cm of 44, fill=white!50!blue] (88) {$\textbf{4}$};
                \node[state, above=5.6cm of 66, fill=orange] (55) {$\textbf{3}$};
                \node[state, above=5.6cm of 88, fill=black!20!green] (77) {$\textbf{6}$};
                \node[state, accepting, below right=4cm and 4.1cm of 33] (9) {$9$};
                \coordinate[above=1.5cm of 0] (0_1);
                \coordinate[below=1.5cm of 0] (0_2);
                \coordinate[above=1.5cm of 9] (3_9);
                \coordinate[below=1.5cm of 9] (4_9);
                \coordinate[below left=1.8cm and 0.3mm of 55] (5_6);
                \coordinate[above right=1.8cm and 0.3mm of 88] (8_7);


                \node[rectangle callout, draw, align=left, callout absolute pointer={($(0_1)-(0)$)}, above left=3cm and 2cm of 0] {\footnotesize\texttt{push(}\circledtext[orange]{1}\texttt{)}};
                \node[rectangle callout, draw, align=left, callout absolute pointer={($(4_9)-(0)$)}, below right=3cm and 3cm of 9] {\footnotesize\texttt{if top == }\circledtext[white!20!red]{5}\texttt{:}\\[-1mm]\footnotesize\texttt{\hspace*{1em}pop(}\circledtext[white!20!red]{5}\texttt{)}};
                \node[rectangle callout, draw, align=left, callout absolute pointer={($(0_2)-(0)$)}, below left=3cm and 2cm of 0] {\footnotesize\texttt{push(}\circledtext[white!50!blue]{2}\texttt{)}};
                \node[rectangle callout, draw, align=left, callout absolute pointer={($(3_9)-(0)$)}, above right=3cm and 3cm of 9] {\footnotesize\texttt{if top == }\circledtext[black!20!green]{6}\texttt{:}\\[-1mm]\footnotesize\texttt{\hspace*{1em}pop(}\circledtext[black!20!green]{6}\texttt{)}};

                \node[rectangle callout, draw, align=left, callout absolute pointer={($(5_6)-(0)$)}, below left=2.8cm and 3cm of 55] {\footnotesize\texttt{if top == }\circledtext[orange]{1}\texttt{:}\\[-1mm]\footnotesize\texttt{\hspace*{1em}pop(}\circledtext[orange]{1}\texttt{)}\\[-1mm]\footnotesize\texttt{\hspace*{1em}push(}\circledtext[white!20!red]{5}\texttt{)}};
                \node[rectangle callout, draw, align=left, callout absolute pointer={($(8_7)-(0)$)}, above right=2.8cm and 3cm of 88] {\footnotesize\texttt{if top == }\circledtext[white!50!blue]{2}\texttt{:}\\[-1mm]\footnotesize\texttt{\hspace*{1em}pop(}\circledtext[white!50!blue]{2}\texttt{)}\\[-1mm]\footnotesize\texttt{\hspace*{1em}push(}\circledtext[black!20!green]{6}\texttt{)}};


                \draw[->, line width=1.5pt, shorten >=1pt] (0) .. controls +(north:3cm) and +(west:2.7cm) .. (11);
                \draw[->, line width=1.5pt, shorten >=1pt] (0) .. controls +(south:3cm) and +(west:4.7cm) .. (22);
                \draw[->, line width=1.5pt, shorten >=1pt] (33) .. controls +(east:2.7cm) and +(north:3cm) .. (9);
                \draw[->, line width=1.5pt, shorten >=1pt] (44) .. controls +(east:4.7cm) and +(south:3cm) .. (9);
                \path[->, line width=1.5pt]
                          (55) edge[] node[left] {} (66)
                          (88) edge[] node[right] {} (77);
                \end{tikzpicture}
        \end{tikzfigure}
    }
\end{columns}

%%%%%%%%%%%%%%%%%%%%%%%%%%%%%%%%%%%%%%%%%%%%
%%%%%%%%%%%%%%%%%%%%%%%%%%%%%%%%%%%%%%%%%%%%
% EXPERIMENTS
%%%%%%%%%%%%%%%%%%%%%%%%%%%%%%%%%%%%%%%%%%%%
%%%%%%%%%%%%%%%%%%%%%%%%%%%%%%%%%%%%%%%%%%%%
\begin{columns}
    \column{0.5}
    \block{Parametric Regular Expressions}
    {
        We evaluated the reduction potential of procedures on 3'656 automata, with an average of 207 states and 2'584 transitions, generated from parametric regular expressions \cite{Regex}.

        \begin{tikzfigure}[\textsc{The reduction ratios achieved by applying procedures to RABIT/Reduce results. On average, procedures improved reductions by 50.3\% in states and 47.9\% in transitions.}\par]
            \centering
            \begin{minipage}{0.21\textwidth}
                \centering
                \includegraphics[width=1\linewidth]{images/intersect-all-states.pdf}
            \end{minipage}
            \begin{minipage}{0.21\textwidth}
                \centering
                \includegraphics[width=1\linewidth]{images/intersect-all-trans.pdf}
            \end{minipage}
            \label{fig:regex}
        \end{tikzfigure}

        The standalone usage of RABIT/Reduce resulted on average in a reduction of 52.5\% in states and 48.4\% in transitions. The further reduction performed by our algorithm can be seen in Figure \ref{fig:regex}. The application of procedures reduced the automata to half the size given by RABIT/Reduce.
    }

    \column{0.5}
    \block[
        bodyoffsety=9cm,
        titleoffsety=9cm
      ]{Network Intrusion Detection System} {

        To test the reduction capability of procedures in a real-world scenario, we used rules from Snort (a well-known NIDS). We generated seven automata, each representing a union of regular expressions from a single category of Snort rules.

        \begin{tikztable}[\textsc{Reduction results of RABIT/Reduce (RAB) and procedures (Proc) on seven sets of Snort rules. $Q$ denotes the number of states $\delta$ the number of transitions, and $\Gamma$ the number of stack symbols. The percentages refer to the results of RABIT/Reduce.}\par]
            \centering
            \small
            \begin{tabular}{c|rr|rr|crrrr}
                Snort rules & \multicolumn{1}{c}{$Q_{in}$} & \multicolumn{1}{c|}{$\delta_{in}$} & \multicolumn{1}{c}{$Q_{RAB}$} & \multicolumn{1}{c|}{$\delta_{RAB}$} & \multicolumn{2}{c}{$Q_{Proc} + \Gamma_{Proc}$} & \multicolumn{2}{c}{$\delta_{Proc}$}\\
                \specialrule{.1em}{.05em}{.05em}
                p2p              & \hspace*{0.5em}33\hspace*{0.5em} &  1'090\hspace*{0.5em}   &\hspace*{0.5em} 32\hspace*{0.5em}   & 1'084\hspace*{0.5em}  & 25+6                   & (96.9\%) & \hspace*{0.5em}570    & (52.6\%)\\
                worm             & \hspace*{0.5em}50\hspace*{0.5em} &  3'880\hspace*{0.5em}   &\hspace*{0.5em} 34\hspace*{0.5em}     & 290\hspace*{0.5em}    & 24+8                  & (94.1\%)& \hspace*{0.5em}284    & (97.9\%)\\
                shellcode        & \hspace*{0.5em}162\hspace*{0.5em} & 3'328\hspace*{0.5em}   &\hspace*{0.5em} 56\hspace*{0.5em}     & 579\hspace*{0.5em}    & 48+2                 & (89.3\%) & \hspace*{0.5em}486    & (83.9\%)\\
                \rowcolor{yellow}mysql            & \hspace*{0.5em}235\hspace*{0.5em} & 30'052\hspace*{0.5em} & \hspace*{0.5em} 91\hspace*{0.5em}  & 14'430\hspace*{0.5em} & 45+18\hspace*{-0.5em}   & (69.2\%) & \hspace*{0.5em}7'142  & (49.5\%)\\
                chat             & \hspace*{0.5em}408\hspace*{0.5em} & 23'937\hspace*{0.5em} & \hspace*{0.5em}113\hspace*{0.5em}   & 1'367\hspace*{0.5em}  & 71+25\hspace*{-0.5em}  & (76.7\%) & \hspace*{0.5em}1'058  & (77.4\%)\\
                \rowcolor{yellow}specific-threats & \hspace*{0.5em}459\hspace*{0.5em} & 57'292\hspace*{0.5em} & \hspace*{0.5em}236\hspace*{0.5em}  & 31'935\hspace*{0.5em} & 99+32\hspace*{-0.5em}   & (55.5\%) & \hspace*{0.5em}12'680 & (39.7\%)\\
                telnet           & \hspace*{0.5em}829\hspace*{0.5em} & 7'070\hspace*{0.5em}  & \hspace*{0.5em}309\hspace*{0.5em}   & 2'898\hspace*{0.5em}  & 155+82                 & (50.0\%) & \hspace*{0.5em}2'164  & (74.7\%)\\
            \end{tabular}
            \normalsize
            \label{table:snort}
        \end{tikztable}

        Among the reduction results in Table~\ref{table:snort}, we highlighted the two most significant reductions. The best size reduction was achieved on the \texttt{specific-threats} rule. RABIT/Reduce tool reduced the automaton by 48.6\% of states and by 44.3\% of transitions. Further application of procedures resulted in an additional reduction of 43.5\% in states and 60.3\% in transitions. This experiment showed that procedures can achieve significant reductions even in real-world examples.
    }
    \block{References} {
        \vspace*{-2em}
        \small

        \bibliographystyle{enplain.bst}
        \bibliography{bibliography}
        \normalsize
    }
\end{columns}

%%%%%%%%%%%%%%%%%%%%%%%%%%%%%%%%%%%%%%%%%%%%%%%%%%%%%%%%%%%%%%%%%%%%%%%%%%%%%%%
%%%%%%%%%%%%%%%%%%%%%%%%%%%%%%%%%%%%%%%%%%%%%%%%%%%%%%%%%%%%%%%%%%%%%%%%%%%%%%%
%%%%%%%%%%%%%%%%%%%%%%%%%%%%%%%%%%%%%%%%%%%%%%%%%%%%%%%%%%%%%%%%%%%%%%%%%%%%%%%
% DOCUMENT END
%%%%%%%%%%%%%%%%%%%%%%%%%%%%%%%%%%%%%%%%%%%%%%%%%%%%%%%%%%%%%%%%%%%%%%%%%%%%%%%
%%%%%%%%%%%%%%%%%%%%%%%%%%%%%%%%%%%%%%%%%%%%%%%%%%%%%%%%%%%%%%%%%%%%%%%%%%%%%%%
%%%%%%%%%%%%%%%%%%%%%%%%%%%%%%%%%%%%%%%%%%%%%%%%%%%%%%%%%%%%%%%%%%%%%%%%%%%%%%%
\end{document}
