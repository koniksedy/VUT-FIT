% dokumentace.tex
% Hlavní tex soubor pro dokumentaci k projektu do předmětu GAL
% Autoři: Bc. Jan Bíl
%         Bc. Michal Šedý
% Poslední úprava: 30.10.2022


% Byla využita šablona šablona pro Bakalářské práce
%==============================================================================

% Autoři / Authors: 2008 Michal Bidlo, 2019 Jaroslav Dytrych
% Kontakt pro dotazy a připomínky: sablona@fit.vutbr.cz
% Contact for questions and comments: sablona@fit.vutbr.cz
%==============================================================================
% nobib - bez literatury
% english - anglický text
% print - černobílé
% cprint - černobílé a barevné obrázky
\documentclass[]{fitthesis} % česky, barevné, s literaturou
%---rm---------------
\renewcommand{\rmdefault}{lmr}%zavede Latin Modern Roman jako rm / set Latin Modern Roman as rm
%---sf---------------
\renewcommand{\sfdefault}{qhv}%zavede TeX Gyre Heros jako sf
%---tt------------
\renewcommand{\ttdefault}{lmtt}% zavede Latin Modern tt jako tt

% vypne funkci šablony, která automaticky nahrazuje uvozovky,
% aby nebyly prováděny nevhodné náhrady v popisech API apod.
% disables function of the template which replaces quotation marks
% to avoid unnecessary replacements in the API descriptions etc.
\csdoublequotesoff

% Balíky
% ======================================================================
\usepackage{url}
\usepackage{lipsum}
\usepackage{amsthm}
\usepackage[ruled,linesnumbered]{algorithm2e}

% Makra
% ======================================================================
\newcounter{DTcounter}
\counterwithin{DTcounter}{section}
\newenvironment{theorem}
  {
    \refstepcounter{DTcounter}
    \stepcounter{figure}
    \stepcounter{equation}
    \bigskip
    \noindent\textbf{Teorém~\theDTcounter}\begin{itshape}
      }
      {
      \end{itshape}
      \leavevmode
  }
\newenvironment{definition}
  {
      \refstepcounter{DTcounter}
      \stepcounter{figure}
      \stepcounter{equation}
      \bigskip
      \noindent\textbf{Definice~\theDTcounter}\begin{itshape}
  }
  {
      \end{itshape}
      \leavevmode
  }
\newenvironment{lemma}
  {
      \refstepcounter{DTcounter}
      \stepcounter{figure}
      \stepcounter{equation}
      \bigskip
      \noindent\textbf{Lemma~\theDTcounter}\begin{itshape}
  }
  {
      \end{itshape}
      \leavevmode
  }

% =======================================================================
% balíček "hyperref" vytváří klikací odkazy v pdf, pokud tedy použijeme pdflatex
% problém je, že balíček hyperref musí být uveden jako poslední, takže nemůže
% být v šabloně
% "hyperref" package create clickable links in pdf if you are using pdflatex.
% Problem is that this package have to be introduced as the last one so it
% can not be placed in the template file.
\ifWis
\ifx\pdfoutput\undefined % nejedeme pod pdflatexem / we are not using pdflatex
\else
  \usepackage{color}
  \usepackage[unicode,colorlinks,hyperindex,plainpages=false,pdftex]{hyperref}
  \definecolor{hrcolor-ref}{RGB}{223,52,30}
  \definecolor{hrcolor-cite}{HTML}{2F8F00}
  \definecolor{hrcolor-urls}{HTML}{092EAB}
  \hypersetup{
	linkcolor=hrcolor-ref,
	citecolor=hrcolor-cite,
	filecolor=magenta,
	urlcolor=hrcolor-urls
  }
  \def\pdfBorderAttrs{/Border [0 0 0] }  % bez okrajů kolem odkazů / without margins around links
  \pdfcompresslevel=9
\fi
\else % pro tisk budou odkazy, na které se dá klikat, černé / for the print clickable links will be black
\ifx\pdfoutput\undefined % nejedeme pod pdflatexem / we are not using pdflatex
\else
  \usepackage{color}
  \usepackage[unicode,colorlinks,hyperindex,plainpages=false,pdftex,urlcolor=black,linkcolor=black,citecolor=black]{hyperref}
  \definecolor{links}{rgb}{0,0,0}
  \definecolor{anchors}{rgb}{0,0,0}
  \def\AnchorColor{anchors}
  \def\LinkColor{links}
  \def\pdfBorderAttrs{/Border [0 0 0] } % bez okrajů kolem odkazů / without margins around links
  \pdfcompresslevel=9
\fi
\fi
% Řešení problému, kdy klikací odkazy na obrázky vedou za obrázek
% This solves the problems with links which leads after the picture
\usepackage[all]{hypcap}

\clubpenalty=10000
\widowpenalty=10000

% checklist
\newlist{checklist}{itemize}{1}
\setlist[checklist]{label=$\square$}

\begin{document}
  % Vysazeni titulnich stran / Typesetting of the title pages
  % ----------------------------------------------

  \begin{titlepage}
    \begin{center}
      \includegraphics[width=0.77 \linewidth]{template-fig/fit-cz.png}

      \vspace{\stretch{1}}

      \Huge{\textbf{Porovnání algoritmů hledání cyklů v grafech}}

      \vspace{\stretch{1}}
    \end{center}


    \begin{minipage}[b]{0.96\textwidth}
      \Large Bc. Jan Bíl  \\
      Bc. Michal Šedý  \hfill \today
    \end{minipage}

  \end{titlepage}

  \setcounter{tocdepth}{2}\setcounter{page}{1}
  % Obsah
  % ----------------------------------------------
  \setlength{\parskip}{0pt}

  {\hypersetup{hidelinks}\tableofcontents}

  % Seznam obrazku a tabulek (pokud prace obsahuje velke mnozstvi obrazku, tak se to hodi)
  % List of figures and list of tables (if the thesis contains a lot of pictures, it is good)
  \ifczech
    \renewcommand\listfigurename{Seznam obrázků}
  \fi
  \ifslovak
    \renewcommand\listfigurename{Zoznam obrázkov}
  \fi
  % {\hypersetup{hidelinks}\listoffigures}

  \ifczech
    \renewcommand\listtablename{Seznam tabulek}
  \fi
  \ifslovak
    \renewcommand\listtablename{Zoznam tabuliek}
  \fi
  % {\hypersetup{hidelinks}\listoftables}

  \ifODSAZ
    \setlength{\parskip}{0.5\bigskipamount}
  \else
    \setlength{\parskip}{0pt}
  \fi

  % vynechani stranky v oboustrannem rezimu
  % Skip the page in the two-sided mode
  \iftwoside
    \cleardoublepage
  \fi

  % Text prace / Thesis text
  % ----------------------------------------------
  % dokumentace-kapitoly.tex
% Tex soubor s jednotlivými kapitolami pro dokumentaci k projektu do předmětu GAL
% Autoři: Bc. Jan Bíl
%         Bc. Michal Šedý
% Poslední úprava: 30.10.2022

\chapter{Úvod}
    \lipsum[2-4]

    Nějaká citace\cite{appleby}.

\chapter{Prerekvizity}
    \lipsum

\chapter{Algoritmus 1}
    \lipsum

\chapter{Algoritmus 2}
    \lipsum

\chapter{Algoritmus 3}
    \lipsum

\chapter{Návrh programu}
    \lipsum

\chapter{Použití programu}
    \lipsum

\chapter{Experimenty}
    \lipsum

\chapter{Závěr}
    \lipsum[2]



  % Pouzita literatura / Bibliography
  % ----------------------------------------------

  \ifczech
    \makeatletter
    \def\@openbib@code{\addcontentsline{toc}{chapter}{Literatura}}
    \makeatother
    \bibliographystyle{bib-styles/Pysny/czplain}
  \else
    \makeatletter
    \def\@openbib@code{\addcontentsline{toc}{chapter}{Bibliography}}
    \makeatother
    \bibliographystyle{bib-styles/Pysny/enplain}
  %  \bibliographystyle{alpha}
  \fi

  \ifbib
    \begin{flushleft}
    \bibliography{dokumentace-bibliografie}
    \end{flushleft}
  \fi

  % vynechani stranky v oboustrannem rezimu
  % Skip the page in the two-sided mode
  \iftwoside
    \cleardoublepage
  \fi

  % Prilohy / Appendices
  % ---------------------------------------------
  \appendix
\ifczech
  \renewcommand{\appendixpagename}{Přílohy}
  \renewcommand{\appendixtocname}{Přílohy}
  \renewcommand{\appendixname}{Příloha}
\fi

  \startcontents[chapters]
  \setlength{\parskip}{0pt}

  \ifODSAZ
    \setlength{\parskip}{0.5\bigskipamount}
  \else
    \setlength{\parskip}{0pt}
  \fi

  % vynechani stranky v oboustrannem rezimu
  \iftwoside
    \cleardoublepage
  \fi

  % Přílohy / Appendices
  % dokumentace-prilohy.tex
% Tex soubor s přílohami pro dokumentaci k projektu do předmětu GAL
% Autoři: Bc. Jan Bíl
%         Bc. Michal Šedý
% Poslední úprava: 30.10.2022


\end{document}
