\documentclass[a4paper, 11pt, fleqn]{scrartcl}

\usepackage[czech]{babel}
\usepackage[utf8]{inputenc}
\usepackage[T1]{fontenc}
\usepackage{times}
\usepackage[left=2cm, top=3cm, text={17cm, 24cm}]{geometry}
\usepackage[unicode, colorlinks, hypertexnames=false, citecolor=red]{hyperref}
\usepackage{fancyhdr}
\usepackage{lastpage}
\usepackage[shortlabels]{enumitem}
\usepackage{amssymb}
\usepackage{amsmath}
\usepackage{newtxtext, newtxmath}
\usepackage{graphicx}
\usepackage{bytefield}
\usepackage[ruled,linesnumbered]{algorithm2e}
\usepackage{multirow}
\usepackage{booktabs}
\usepackage{caption}

%%%%%%%%%%%%%%%%%%%%%%%%%%%%%%%%%%%%%%%%%%%%%%%%%%%%%%%%%%%%%%%%%%%%%
% Makra %%%%%%%%%%%%%%%%%%%%%%%%%%%%%%%%%%%%%%%%%%%%%%%%%%%%%%%%%%%%%

\newcommand{\KURZ}{Kódování a Komprese Dat}
\newcommand{\AUTOR}{Michal Šedý (xsedym02)}
\newcommand{\NAZEV}{\textbf{Komprese obrazových dat s využitím Huffmanova kódování}}

%%%%%%%%%%%%%%%%%%%%%%%%%%%%%%%%%%%%%%%%%%%%%%%%%%%%%%%%%%%%%%%%%%%%%
% Hlavička %%%%%%%%%%%%%%%%%%%%%%%%%%%%%%%%%%%%%%%%%%%%%%%%%%%%%%%%%%

\pagestyle{fancy}
\fancyhead[L]{\AUTOR}
\fancyhead[C]{\KURZ}
\fancyhead[R]{\today}

\fancyfoot[C]{}
\fancyfoot[R]{\thepage\,/\,\pageref*{LastPage}}

\setlength{\parindent}{0pt}
\setlength{\mathindent}{0pt}

%%%%%%%%%%%%%%%%%%%%%%%%%%%%%%%%%%%%%%%%%%%%%%%%%%%%%%%%%%%%%%%%%%%%%
% Text %%%%%%%%%%%%%%%%%%%%%%%%%%%%%%%%%%%%%%%%%%%%%%%%%%%%%%%%%%%%%%

\begin{document}

  \begin{center}
    {\Large \NAZEV}
  \end{center}

  \vspace*{2em}

    Tato dokumentace popisuje kompresní program \texttt{huff\_codec} implementovaný v~jazyce C++ do předmětu kódování a komprese dat. Program slouží pro kódování a dekódování obrázků v odstínech šedi s využitím Huffmanova kanonického kódu. Pro zvýšení kompresního výkonu je uplatněno adaptivní skenování po blocích o velikosti 64x64 bitů a dále jsou využity čtyři modely pracující s rozdílem sousedících pixelů. Tento projekt byl vypracován výhradně na základě informací uvedených na přednáškách.

  \section{Model}
    V programu jsou využity čtyři modely pracující s rozdílem hodnot sousedních pixelů. Byly implementovány modely procházející obraz po řádcích z leva do prava, po sloupcích shora dolů, paralelně s hlavní diagonálou shora dolů a paralelně s vedlejší diagonálou shora dolů. Kompresní výkon modelu je aproximován velikostí abecedy modelu (počtem rozdílných hodnot). Pro kódování je využit pouze model s nejmenší abecedou. Pro použití modelu během kódování programem je nutné zadat přepínač \texttt{-m}.

    \subsection{Po řádcích}
      Při průchodu matice bodů $X$ po řádcích zleva doprava je za vztažný bod zvolen levý horní bod $x_{0,0}$. Rozdíly sousedících bodů jsou vypočítány pomocí následujícího předpisu.
      $$
      y_{i,j} = \begin{cases}
                  x_{i,j} - x_{i,j-1}   & j > 0\\
                  x_{i,j} - x_{i-1,j}   & j = 0 \land i > 0\\
                  0                     & jinak
                \end{cases}
      $$

    \subsection{Po sloupcích}
      Při průchodu matice bodů $X$ po sloupcích shora dolů je ze vztažný bod zvolen levý horní bod $x_{0,0}$. Rozdíly sousedících bodů jsou vypočítány pomocí následujícího předpisu.
      $$
      y_{i,j} = \begin{cases}
                  x_{i,j} - x_{i-1,j}   & i > 0\\
                  x_{i,j} - x_{i,j-1}   & i = 0 \land j > 0\\
                  0                     & jinak
                \end{cases}
      $$

    \subsection{Paralelně s hlavní diagonálou}
      Při průchodu matice bodů $X$ paralelně s hlavní diagonálou z levého horního bodu ke spodnímu pravému je za vztažný bod zvolen levý horní bod $x_{0,0}$. Rozdíly sousedících bodů jsou vypočítány pomocí následujícího předpisu.
      $$
      y_{i,j} = \begin{cases}
                  x_{i,j} - x_{i-1,j-1}   & i > 0 \land j > 0\\
                  x_{i,j} - x_{i-1,j}     & i > 0 \land j = 0\\
                  x_{i,j} - x_{i,j-1}     & i = 0 \land j > 0\\
                  0                       & jinak
                \end{cases}
      $$

    \subsection{Paralelně s vedlejší diagonálou}
      Při průchodu matice bodů $X$ paralelně s vedlejší diagonálou z pravého horního bodu ke spodnímu levému je za vztažný bod zvolen pravý horní bod $x_{n,n}$. Rozdíly sousedících bodů jsou vypočítány pomocí následujícího předpisu.
      $$
      y_{i,j} = \begin{cases}
                  x_{i,j} - x_{i-1,j+1}   & i > 0 \land j < n\\
                  x_{i,j} - x_{i-1,j}     & i > 0 \land j = n\\
                  x_{i,j} - x_{i,j+1}     & i = 0 \land j < n\\
                  0                       & jinak
                \end{cases}
      $$

    \section{Adaptivní skenování}
      Velikost vstupní abecedy lze snížit s využitím adaptivního skenování, které rozdělí matici dat na menší bloky o~velikosti 64x64 bitů. Během testů se ukázalo, že menší velikost bloku než 64x64 bitů nezvyšuje kompresní výkon, naopak zvyšuje velikost hlaviček v komprimovaných datech. Pro využití adaptivního skenování v programu je nutné zadat přepínač \texttt{-a}.

    \section{Kanonický Huffmanův kód}
      Táto sekce popisuje využití kanonického Huffmanova kódu pro kódování a dekódování obrazových dat.

      \subsection{Kódování}
        Pro kódování dat v matici $X$ s abecedou $\Sigma_{EOF} = \Sigma \cup \{\textit{EOF}\}$, kde $\Sigma$ je abeceda matice $X$ a $\textit{EOF}$ je speciální znak, který se v $\Sigma$ nevyskytuje, byl využit kanonický Huffmanův kód, který byl získán transformací stromu $T$ reprezentujícího Huffmanův kód pro $\Sigma_{EOF}$. Listový uzel $n \in leaves(T)$ stromu $T$ obsahuje znak $a \in \Sigma_{EOF}$, informaci o jeho četnosti výskytu v matici $X$ ($cnt(a) \in \mathbb{N}$), kde $cnt(\textit{EOF}) = 1$, a o hloubce uzlu ve stromu ($\textit{depth}(n) \in \mathbb{N}$).

        \begin{algorithm}[!h]
          \SetAlgoLined
          \DontPrintSemicolon
          \caption{Konstrukce stromu Huffmanova kódu}
          \KwIn{matice dat $X$, abeceda matice $\Sigma_{EOF}$}
          \KwOut{strom $T$ Huffmanova kódu}

          \vspace*{2mm}

          $pQueue \leftarrow \textit{reversePriorityQueue()}$ \tcp*{uzly s nejmenším $cnt$ jsou na čele}
          \ForEach{$a \in \Sigma_{EOF}$}
          {
            $node \leftarrow Node()$\tcp*{vytvoř nový uzel}
            $node.symbol \leftarrow a$\;
            $node.cnt \leftarrow cnt(a)$\;
            $pQueue.put(node)$\;
          }

          \vspace*{1mm}

          \While{$|pQueue| > 1$}
          {
            $n \leftarrow pQueue.pop()$\;
            $m \leftarrow pQueue.pop()$\;
            $node \leftarrow Node()$\tcp*{vytvoř nový uzel}
            $node.cnt \leftarrow n.cnt + m.cnt$\;
            $pQueue.put(node)$\;
          }

          \Return{$pQueue.pop()$}

        \end{algorithm}

        \newpage

        \enlargethispage{1\baselineskip}

        \vspace*{-1.2cm}
        \begin{algorithm}[!h]
          \SetAlgoLined
          \DontPrintSemicolon
          \caption{Konstrukce kanonického Huffmanova kódu}
          \KwIn{matice dat $X$, abeceda matice $\Sigma$}
          \KwOut{kódovací funkce $\textit{huff}: \Sigma_{EOF} \rightarrow \mathbb{N}_0$}

          $\Sigma_{EOF} \leftarrow \Sigma \cup \{\textit{EOF}\}$\;
          $\textit{tree} \leftarrow \textit{getHuffmanTree}(X, \Sigma_{EOF})$\;
          $\textit{sortedLeaves} \leftarrow \textit{ascendSort}(\textit{leaves}(\textit{tree}))$\tcp*{podle $depth(node)$ a $(node.symbol)$}

          $n_0 \leftarrow \textit{sortedLeaves}[0]$\;
          $\textit{huff}(n_0.symbol) \leftarrow 0$\;


          $i \leftarrow 1$\;
          \While{$i < |\textit{sortedLeaves}|$}
          {
            $n_{i-1} \leftarrow \textit{sortedLeaves}[i-1]$\;
            $n_i \leftarrow \textit{sortedLeaves}[i]$\;
            $\textit{huff}(n_i.symbol) \leftarrow \textit{huff}(n_{i-1}.symbol + 1) << \textit{depth}(n_i) - \textit{depth}(n_{i-1})$\;
            $i \leftarrow i+1$\;
          }


          \Return{$\textit{huff}$}

        \end{algorithm}

      \vspace*{-0.2cm}

      \subsection*{Dekódování}
        Pro dekódování jsou z hlavičky dat kanonického Huffmanova kódu získány: maximální délka kódového slova $N$, funkce $L: \mathbb{N} \rightarrow \mathbb{N}_0$, kde $L(i)$ udává počet kódových slov délky $i$, a kódová abeceda $\Sigma$ s uspořádáním odpovídajícím $\textit{sortedLeaves}$ z předchozího algoritmu. Dekódování dat je prováděno přímo bez konstrukce binárního stromu.


        \begin{algorithm}[!h]
          \SetAlgoLined
          \DontPrintSemicolon
          \caption{Dekódování kanonického Huffmanova kódu}
          \KwIn{vstupní proud dat $inStream$, maximální délka kódového slova $N$, $L: \mathbb{N} \rightarrow \mathbb{N}_0$, $\Sigma$}
          \KwOut{dekódovaná data $out$}

          $c \leftarrow 0$\;
          $s \leftarrow 1$\;
          \For{$i = 0 \dots N$}
          {
            $\textit{firstCode}(i) \leftarrow c$\;
            $\textit{firstSymbol}(i) \leftarrow s$\;
            $s \leftarrow s + L(i)$\;
            $c \leftarrow (c + L(i)) << 1$\;
          }

          $l \leftarrow c \leftarrow 0$\;
          $symbol \leftarrow NaN$\;
          \While{$symbol \neq \textit{EOF}$}
          {
            $c \leftarrow (c << 1) + getBit(inStream)$\;
            \uIf{$(c << 1) < firstCode(l+1)$}
            {
              $symbol \leftarrow \Sigma[firstSymbol(l) + c + firstCode(l) - 1]$\;
              \If{$symbol \neq \textit{EOF}$}
              {
                $out.append(symbol)$\;
              }
              $l \leftarrow c \leftarrow 0$\;
            }
            \Else{
              $ l \leftarrow l + 1$\;
            }
          }

          \Return{$out$}
        \end{algorithm}

    \newpage

    \section{Hlavičky}
      Zakódovaná data obsahují hlavičky dvou úrovní. Nejvyšší úroveň informuje o šířce dekódovaného obrazu (\texttt{Image Width}), zda bylo použito adaptivní skenování (\texttt{Adaptive = 1}) a zda byl použit model (\texttt{Model~=~1}). Za hlavičkou následují data jednotlivých zakódovaných bloků obrazu. Na konci dat se nachází zarovnání.

      \vspace*{2em}

      \begin{figure}[!h]
        \centering
        \hspace*{6mm}
        \begin{bytefield}[rightcurly=., rightcurlyspace=0pt]{16}
          \begin{rightwordgroup}{16b}
            \wordbox{1}{Image Width}
          \end{rightwordgroup} \\
          \begin{rightwordgroup}{1b}
            \wordbox{1}{Adaptive}
          \end{rightwordgroup} \\
          \begin{rightwordgroup}{1b}
            \wordbox{1}{Model}
          \end{rightwordgroup} \\
          \wordbox{2}{Data Blocks}\\
          \begin{rightwordgroup}{0 $\dots$7b}
            \wordbox{1}{Padding}
          \end{rightwordgroup}
        \end{bytefield}
        \caption{Význam jednotlivých bitů v zakódovaném obraze.}
      \end{figure}

      \vspace*{1em}

      Hlavička jednotlivých bloků obsahuje informaci o směru průchodu datovou maticí v případě využití modelu (\texttt{Pass}, kde \texttt{00} = průchod zleva doprava, \texttt{01} = průchod shora dolů, \texttt{10} = průchod paralelně s hlavní diagonálou a \texttt{11} = průchod paralelně s vedlejší diagonálou), referenční bod modelu (\texttt{Ref Point}), maximální délku slova v~kanonickém Huffmanově kódu (\texttt{N}), počet kódových slov délky $i$ ($L_i$ pro $0 < i \leq N$) a symboly abecedy ($\Sigma_0$ až $\Sigma_K$). V případě, že při kódování nebyl využit model, nemají pole \texttt{Pass} a \texttt{Ref Point} význam. Za hlavičkou se nachází surová data určena pro dekódování. Data jsou zakončena zakódovaným znakem \texttt{EOF}.

      \vspace*{2em}

      \begin{figure}[!h]
        \centering
        \begin{bytefield}[rightcurly=., rightcurlyspace=0pt]{16}
          \begin{rightwordgroup}{2b}
            \wordbox{1}{Pass}
          \end{rightwordgroup} \\
          \begin{rightwordgroup}{8b}
            \wordbox{1}{Ref Point}
          \end{rightwordgroup} \\
          \begin{rightwordgroup}{16b}
            \wordbox{1}{N}
          \end{rightwordgroup} \\
          \begin{rightwordgroup}{9b}
            \wordbox{1}{$L_1$}
          \end{rightwordgroup} \\
          \wordbox[]{1}{$\vdots$} \\[1ex]
          \begin{rightwordgroup}{9b}
            \wordbox{1}{$L_N$}
          \end{rightwordgroup} \\
          \begin{rightwordgroup}{9b}
            \wordbox{1}{$\Sigma_0$}
          \end{rightwordgroup} \\
          \wordbox[]{1}{$\vdots$} \\[1ex]
          \begin{rightwordgroup}{9b}
            \wordbox{1}{$\Sigma_K$}
          \end{rightwordgroup} \\
          \wordbox{2}{Huffman Data}
        \end{bytefield}
        \caption{Význam jednotlivých bitů v zakódovaném datovém bloku.}
      \end{figure}

    \newpage

    \shorthandoff{-}
    \section{Experimentální výsledky}
      Testování bylo prováděno na 8 jádrovém procesoru AMD Ryzen 7 3800XT s 32 GB RAM. V následujících tabulkách jsou uvedeny efektivity jednotlivých kompresí: bez využití modelu a bez adaptivního skenování, s~využitím adaptivního skenování (\texttt{-a}), s~využitím modelu (\texttt{-m}) a~s~využitím adaptivního skenování a modelu (\texttt{-a -m}). Tabulka 1 udává výsledky pro jednotlivé soubory ze zadání. Tabulka 2 uvádí průměrnou efektivitu kompresí v rámci všech souborů.

      \begin{table}[!h]
        \centering
        \captionsetup{justification=justified}
        \begin{tabular}{|c|c||c|c|c|c|c|c|c|c|c|c|}
          \cline{3-12}
          \multicolumn{2}{c||}{}                                 & df1h & df1hvx & df1v & hd01 & hd02 & hd07 & hd08 & hd09 & hd12 & nk01\\[0.5ex]
          \cline{2-12}\morecmidrules\cmidrule{2-12}
          \multicolumn{1}{c|}{}       & Entropie                & 8.00 &  4.51  & 8.00 & 3.83 & 3.64 & 5.58 & 4.21 & 6.62 & 6.17 & 6.47\\[0.5ex]
          \hline
          \multirow{4}{*}{Efektivita} &  \texttt{NaN}  & 8.01 &  4.58  & 8.01 & 3.88 & 3.70 & 5.61 & 4.23 & 6.66 & 6.20 & 6.50\\[0.5ex]
                                      & \texttt{-a}    & 6.18 &  3.91  & 6.18 & 3.87 & 3.77 & 4.64 & 3.84 & 5.59 & 5.41 & 6.28\\[0.5ex]
                                      & \texttt{-m}    & 1.00 &  1.96  & 1.00 & 3.41 & 3.34 & 3.75 & 3.29 & 4.50 & 4.19 & 5.48\\[0.5ex]
                                      & \texttt{-a -m} & 1.04 &  1.54  & 1.03 & 3.40 & 3.35 & 3.38 & 3.20 & 4.30 & 3.85 & 5.62\\[0.5ex]
          \hline
          \multirow{4}{*}{Čas [s]}    & \texttt{NaN}   & .026 &  .016  & .022 & .020 & .020 & .024 & .020 & .027 & .025 & .029\\[0.5ex]
                                      & \texttt{-a}    & .021 &  .016  & .021 & .024 & .024 & .026 & .021 & .031 & .029 & .033\\[0.5ex]
                                      & \texttt{-m}    & .014 &  .021  & .014 & .036 & .037 & .036 & .034 & .042 & .037 & .051\\[0.5ex]
                                      & \texttt{-a -m} & .017 &  .020  & .017 & .041 & .043 & .038 & .040 & .048 & .044 & .059\\[0.5ex]
          \hline

        \end{tabular}
        \caption{Tabulka uvádí výsledky komprese s využitím kombinací adaptivního skenování (\texttt{-a}) a modelu (\texttt{-m}). Efektivita komprese udává průměrný počet bitů potřebných k zakódování jednoho pixelu.}
      \end{table}

      \begin{table}[!h]
        \centering
        \captionsetup{justification=justified}
        \begin{tabular}{|c|c|}
          \hline
          Přepínače & Efektivita\\[0.5ex]
          \hline\hline
          \texttt{NaN}   & 4.78\\[0.5ex]
          \texttt{-a}    & 4.14\\[0.5ex]
          \texttt{-m}    & 2.66\\[0.5ex]
          \texttt{-a -m} & 2.39\\[0.5ex]
          \hline
        \end{tabular}
        \caption{Tabulka uvádí průměrnou efektivity pro různé kombinace adaptivního skenování (\texttt{-a}) a modelu (\texttt{-m}).}
      \end{table}

    \section{Překlad a spuštění}
      Tato sekce poskytuje základní informace o sestavení projektu, jeho spuštění a automatickém testování.

      \subsection{Překlad}
        Po zadání příkazu \texttt{make} v kořenovém adresáři projektu dojde k přeložení a vytvoření spustitelného souboru \texttt{huff\_codec}. Příkazem \texttt{make clean} pak jsou odstraněny všechny soubory vytvořené během překladu, a také soubor \texttt{huff\_codec}.

      \subsection{Spuštění}
        Program lze spustit v režimu kódování s přepínačem \texttt{-c}, nebo dekódování \texttt{-d}. Při kódování je nutné zadat šířku obrazu (\texttt{-w}). Šířka musí být s intervalu $\langle 1; 65535 \rangle$. Pro kódování lze použít model (\texttt{-m}) a adaptivní skenování (\texttt{-a}). Programu je nutné vždy specifikovat vstupní (\texttt{-i}) a výstupní (\texttt{-o}) soubory.

      \subsection{Automatické testy}
        Za účelem testování kompresního programu byly vytvořeny automatické testy, které se nacházejí ve složce \texttt{test}. Automatické testy lze spustit z kořenové složky projektu zadáním příkazu \texttt{make test}.

\end{document}
