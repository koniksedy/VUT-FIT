%\documentclass{PPFIT} % pro psaní v angličtině / for writing in English
\documentclass[czech]{PPFIT} % pro psaní v češtině / for writing in Czech
%\documentclass[slovak]{PPFIT} % pro psaní ve slovenštině / for writing in Slovak

%--------------------------------------------------------
%--------------------------------------------------------
%	PŘIZPŮSOBENÍ PDF / PDF CUSTOMIZATION
%--------------------------------------------------------
\usepackage{stfloats}
\usepackage{lipsum}

\hypersetup{
	pdftitle={Využití opakujících se podstruktur pro efektivní reprezentaci automatů},
	pdfauthor={Bc. Michal Šedý},
	pdfkeywords={nedeterministický konečný automaty, nedeterministický zásobníkový automaty, minimalizace, simulace}
}

%--------------------------------------------------------
%--------------------------------------------------------
%  PŘEDBĚŽNÁ vs. KONEČNÁ VERZE / REVIEW vs. FINAL VERSION
%--------------------------------------------------------
%   NECHTE následující řádek zakomentovaný pro PŘEDBĚŽNÉ VERZE
%   Pro KONEČNOU VERZI řádek odkomentujte.
%   LEAVE this line commented out for the REVIEW VERSIONS
%   UNCOMMENT this line to get the FINAL VERSION

\PPFinalCopy


\PPYear{2022/2023}

\ifczechslovak%
    %--------------------------------------------------------
    %--------------------------------------------------------
    %	INFORMACE O ČLÁNKU
    %--------------------------------------------------------

  \PaperTitle{Hry s nepřenositelným užitkem}

	\Authors{Bc. Michal Šedý}
	\affiliation{%
	  \href{mailto:xsedym02@stud.fit.vutbr.cz}{xsedym02@stud.fit.vutbr.cz},
	  \textit{Fakulta informačních technologií, Vysoké učení technické v Brně}}

	\Keywords{kooperativní hry --- hry s přenositelným užitkem --- hry s nepřenositelným užitkem --- jádro hry --- Shapleyho hodnota}

	% \Supplementary{\href{http://youtu.be/S3msCdn3fNM}{Demonstrační Video} --- \href{http://fit.vut.cz/}{Stáhnutelný Kód}}

    %--------------------------------------------------------
    %--------------------------------------------------------
    %	ABSTRAKT
    %--------------------------------------------------------
    \Abstract{
        Tato práce popisuje hry z nepřenositelným užitkem (NTU--games), převody nekooperativních her na NTU--hry a mechanizmy popisující jejich analýzu, jakými jsou jádro hry a Shapleyho NTU--hodnota. Následující text pracuje se superaditivními hrami, pro které platí, že zisk dvou disjunktních koalic je vždy menši, nebo roven ziku, kterého by dosáhla jedna velká koalice sestávající ze členů techto dvou menších koalic. Speciální podmnožinou NTU her jsou hry s přenositelným užitkem (TU--games), kde zatímco TU hry rozdělují mezi hráče libovolně dělitelnou komoditu (bez ztráty na obecnosti si lze tuto komoditu představit jako peníze), tak v případě NTU her si hráči rozdělují množinu nedělitelných komodit. Jedním z příkladu NTU her je směnná ekonomika, ve které hráči utvářejí koalice a redistribuují své počáteční vklady (nejedná se pouze o peníze) za účelem zisku.
    }
\else%
    %--------------------------------------------------------
    %--------------------------------------------------------
    %	INFORMATION ABOUT THE ARTICLE
    %--------------------------------------------------------
    % \PaperTitle{How to write an amazing article}

	% \Authors{Adam Herout*}
	% \affiliation{*%
	% 	\href{mailto:herout@fit.vut.cz}{herout@fit.vut.cz},
	% 	\textit{Faculty of Information Technology, Brno University of Technology}}

	% \Keywords{Keyword1 --- Keyword2 --- Keyword3}

	% \Supplementary{\href{http://youtu.be/S3msCdn3fNM}{Demonstation video} --- \href{http://fit.vut.cz/}{Downloadable code}}

	% %--------------------------------------------------------
    % %--------------------------------------------------------
    % %	ABSTRACT
    % %--------------------------------------------------------
    % \Abstract{}
\fi

%--------------------------------------------------------
%--------------------------------------------------------
%	TEASER
%--------------------------------------------------------
% \Teaser{
% %	\TeaserImage{placeholder.pdf}
% %	\TeaserImage{placeholder.pdf}
% %	\TeaserImage{placeholder.pdf}
% }

%--------------------------------------------------------
%--------------------------------------------------------
%--------------------------------------------------------
%--------------------------------------------------------
\begin{document}
\startdocument
%--------------------------------------------------------
%--------------------------------------------------------
%	OBSAH ČLÁNKU / CONTENT OF THE ARTICLE
%--------------------------------------------------------
\ifczechslovak
    \section{Úvod}
    Kooperativní teorie her se zabývá hrami se dvěma a~více hráči, jejichž spolupráce (dohodnuti vymahatel-né strategie) může vést k lepším ziskům hráčů ze hry, než kdyby hra proběhla nekooperativně. Žádný hráč by neměl obdržet menší zisk, než by mu přineslo odmítnutí spolupráce. V takovém případě by se již nejednalo o~kooperativní hru.

    Cílem této práce je popis kooperativní hry s nepře-nositelným užitkem (NTU--hra) \cite{Coop_intro,Game_Theory,EOLSS}, pro kterou je stanovena množina hráčů, kteří spolu tvoří koalice a~charakteristická funkce, která každé koalici přiřazuje množinu možných výplatních vektorů. Budeme pracovat pouze se superaditivními NTU hrami, pro které platí, že je součet užitků dvou disjunktních koalic menší, nebo roven užitku větší koalice vzniklé slouče-ním těchto dvou menších koalicí. Vyjednávání o utvá-ření koalic lze tedy zanedbat, protože na základě podmínky superaditivity lze vidět, že hráči budou vždy chtít utvořit koalici složenou ze všech přítomných hráčů (velká koalice). Jediným místem vyjednávání v~těchto NTU--hrách je volba výplatního vektoru z~mno-žiny určeného charakteristickou funkcí pro velkou koalici. Pokud jsou hráči racionální, pak žádný z~nich nepřijme výplatní vektor, který by mu poskytoval menší zisk, než by mohl získat v jiné koalici. Množina všech výplatních vektorů koalice, které jsou ochotni její racionální členové přijmout se nazývá jádro hry. Je také potřeba určit "nejspravedlivější" výplatní vektor, který bere v~úvahu přínos každého člena do koalice, tento vektor specifikuje Shapleyho NTU--hodnota \cite{Shapley1969}.

    Mezi dvě nejznámější podtřídy NTU--her patří dvouhráčové kooperativní hry s vyjednávání \cite{Game_Theory}, pro které existuje Nashovo řešení \cite{Nash1950}, druhou podtřídou jsou pak kooperativní hry s přenositelným užitkem (TU--hry) \cite{Game_Theory,EOLSS}. TU--hry přiřazují každé koalici libovolně dělitelnou komoditu (touto komoditou mohou být například peníze), která je "spravedlivě" distribuována mezi členy koalice. Oproti tomu NTU--hry simulují situaci, ve které existuje pouze množina nedělitelných komodit (například zboží ve směnné ekonomice), které jsou přerozděleny mezi členy koalice.

    Z důvodu vysoké podobnosti mezi TU a NTU hrami jsou v sekci \ref{sec:TU} uvedeny základní definice spojené s analýzou TU--her.

    Sekce \ref{sec:NTU} obsahuje definici n-hráčové kooperativní hry s nepřenositelným užitkem, dále sekce uvádí principy převodu nekooperativní strategické hry v normální formě \cite{Hruby2022NCOOP}, dvouhráčové kooperativní hry s vyjednávání, TU--hry a veřejné volby na ekvivalentní NTU--hru. Současně je také uveden opačný převod, kdy je NTU--hra transformována na nekooperativní strategickou hru v normální formě.

    Definice jádra NTU--hry udávajícího pro velkou koalici všechny racionální výplatní vektory je uvedena v kapitole \ref{sec:Core}.

    V poslední sekci \ref{sec:Shapley} je popsána Shapleyho NTU--hodnota (jinak také zvaná $\lambda$--přenosová NTU--hodnota), která specifikuje "spravedlivý" výplatní vektor.


\section{TU--hra}
    \label{sec:TU}
    Tato sekce poskytuje základní definice a pojmy spojené s TU--hrami. V pozdějších částech práce budou právě tyto definice využívány při popisu NTU--her a jejich analýzy. Text této kapitoly je převzata z \cite{Hruby2022}.

    \subsection{Hra s přenositelným užitkem}
        \textit{\textbf{Hra s přenositelným užitkem}} je dvojice $\Gamma = (Q, v)$, kde $Q$ je množina všech hráčů a $v: 2^Q \rightarrow \mathbb{R}$ je charakteristická funkce udávající sílu koalice (množství pro-středků k přerozdělení mezi členy koalice). Platí, že $v(\emptyset) = 0$.

        Koalici $K = Q$ budeme nazývat \textit{velkou koalicí}. Jak již bylo dříve zmíněno, budeme uvažovat hry se superaditivitou. TU-hra $\Gamma$ je \textit{superaditivní}, pokud pro každé dvě disjunktní koalice $K$ a $L$ platí $v(K \cup L) \geq v(K) + v(L)$.

        Pro vyjádření rozdělení zisků $v(K)$ v koalici $K$ zavedeme \textit{výplatní vektor} $a \in \mathbb{R}^{|Q|}$, kde $a(K) = \sum_{i\in K}a_i$. Ve hrách budeme pracovat pouze s \textit{prostorem efektivních zisků} $X^*(v) = \{a \in \mathbb{R}^{|Q|}\,|\,a(Q) = v(Q)\}$.

    \subsection{Imputace v TU--hře}
        Výplatní vektor $a \in X^*(v)$ je \textit{\textbf{individuálně racionální}}, pokud pro všechny hráče $i \in Q$ platí $a_i \geq v(\{i\})$. Tedy žádný hráč nesmí v koalici dostat méně, než kdyby byl v jednočlenné koalici (sám).

        Výplatní vektor $a \in X^*(v)$ je \textit{\textbf{kolektivně racionální}}, pokud $\sum_{i \in Q}a_i = v(Q)$. Každá koalice musí tedy rozdat všechny "zdroje".

        Nechť $\Gamma = (Q, v)$ je TU--hra, kde $N = |Q|$, potom je N-tice $a \in X^*$ \textit{\textbf{imputace}}, pokud jsou pro $a$ splněny podmínky individuální a kolektivní racionality. Prostor všech imputací hry budeme značit $X(v)$.

        Mějme TU--hru $\Gamma = (Q, v)$, koalici $K \subseteq Q$ a dvě imputace $a, b$. Řekneme, že $a$ \textit{\textbf{dominuje}} $b$ pro koalici $K$ (značíme $a \succ_K b$), pokud platí $\forall i \in K: a_i > b_i$ a~zároveň $\sum_{i \in K}a_i \leq v(K)$.

    \subsection{Jádro TU--hry}
        \textit{\textbf{Jádro hry}} $\Gamma = (Q, v)$ je tvořeno množinou imputací $C(v) = \{a \in X(v)\,|\,\sum_{i \in K} \geq v(K); \forall K \in 2^Q \setminus \emptyset\}$. Jedná se o takovou množinu imputací, že každá případná koalice $K$ obdrží alespoň $v(K)$. Pro každou imputaci v~jádře zároveň platí, že není dominována žádnou jinou imputací.

        Pokud je jádro prázdné, pak neexistuje stabilní kooperativní řešení, které by ustanovilo velkou koalici.

    \subsection{Shapleyho hodnota v TU--hře}
        \textit{\textbf{Shapleyho hodnota}} \cite{Shapley1953} $H_i$ modeluje přínos hráče $i \in Q$ do velké koalice $Q$. Na základě velikosti přínosu  volí hráči takovou imputaci, která spravedlivě přiděluje zisky.

        Při neúčasti hráče $i$ v koalici $K$ je ztráta způsobena koalici $K$ rovna $\delta(i, K) = v(K) - v(K \setminus \{i\})$.

        Přínos hráče $i$ do všech $k$-členných koalic je:
        \vspace*{-0.3em}
        $$
        h_i(k) = \sum_{K \subset Q, k = |K|, i \in K}\frac{\delta(i, K)}{{{N-1} \choose {k-1}}}
        $$

        Průměrný přínos hráče $i$ do všech možných koalic velikosti 1 až $|Q|$ je:
        \vspace*{-0.3em}
        $$
        H_i = \sum^N_{k = 1}\frac{h_i(k)}{N}
        $$

        Mějme TU--hru $\Gamma = (Q, v)$, \textit{\textbf{Shapleyho vektor}} této hry je definován funkci $\phi$ jako $\phi(v) = \mathbb{H} = (H_1, H_2, \dots, \\H_N)$. Pro Shapleyho vektor platí, že je imputací hry.

\section{NTU--hra}
    \label{sec:NTU}
    Hry s nepřenositelným užitkem \cite{Coop_intro,Game_Theory,EOLSS} (NTU--hry) neumožňují hráčům v koalici libovolně přerozdělit zisk určený pro celou koalici, jak je tomu u TU--her, namísto toho je charakteristickou funkcí určena mno-žina výplatních vektorů pro koalici. Účastnici koalice si tedy mohou volit pouze s poskytnutých výplatních vektoru.

    Protože se v následujících sekcích bude hojně pracovat s výplatními vektory, připomeneme definice základních vektorových operací, které budou používány. Pro porovnávání dvou vektorů $x, y \in \mathbb{R}^n$ používáme binární operátory $\geqslant$, $\leqslant$, $\ll$, $\gg$, kde $x \leqslant y \iff \forall 1 \leq i \leq n: x_i \leq y_i$. Ostatní operátory jsou definovány obdobně. Skalární součin dvou vektorů $x, y \in \mathbb{R}^n$ je definován $x \cdot y = \sum^n_{i = 1}x_i \cdot y_i$. Výsledkem vektorového součinu dvou vektorů $x, y \in \mathbb{R}^n$ je opět vektor stejné dimenze a je značen $xy = (x_i \cdot y_i)_{i \in \{1, \dots, n\}}$.

    \subsection{Hra s nepřenositelným užitkem}
        \textit{\textbf{Hra s nepřenositelným užitkem}} je dvojice $\Gamma = (Q, V)$, kde $Q$ je množina hráčů, kteří tvoří koalice $K \subseteq Q$ a $V$ je charakteristická funkce tvaru $V: 2^Q \rightarrow \mathbb{R}^{|Q|}$. Pro libovolnou koalici $K \subseteq Q$ odpovídá množina $V(K)$ následujícím axiomům:

        \begin{itemize}
            \item[(A1)] $V(K) \neq \emptyset \iff K \neq \emptyset$,
            \item[(A2)] uzavřená,
            \item[(A3)] konvexní,
            \item[(A4)] pokud pro $x \in V(K)$ a $y \in \mathbb{R}^{|Q|}$ platí $x \geqslant y$, pak $y \in V(K)$,
            \item[(A5)] shora omezená (pro každé $a \in \mathbb{R}^{|Q|}$ je množina $\{a \in V(K)\,|\, y \geqslant x\}$ kompaktní),
            \item[(A6)] v každém bodě z hranice $\partial V(K)$ má podpornou rovinu,
            \item[(A7)] $\forall x, y \in \partial V(K): x \geqslant y \implies x = y$.
        \end{itemize}

        V této práci se zabýváme pouze superaditivními hrami. NTU--hra $\Gamma = (Q, V)$ je \textit{superaditivní}, pokud pro každou dvojici koalic $K, L \in 2^Q \setminus \{\emptyset\}$, kde $K \cap L = \emptyset$, platí $V(K) \times V(L) \subset V(K \cup L)$.

    \subsection{Převod jiných her na NTU--hru}
        Tato část práce ukazuje postupy, jak lze převést strategickou hru, vyjednávají hru dvou hráčů, TU--hru nebo veřejnou volbu na ekvivalentní NTU--hru \cite{Game_Theory}.

        \subsubsection*{Strategická hra n hráčů}
            \textit{\textbf{Strategická hra}} $N$ hráčů \cite{Hruby2022NCOOP} je trojice $\Gamma_S = (Q, (S_i)_{i\in Q},\\ (U_i)_{i\in Q})$, kde:
            \begin{itemize}
                \item $Q = \{1, \dots, N\}$ je množina hráčů,
                \item $S_i$ je množina strategií hráče $i \in Q$,
                \item $U_i: \prod_{j\in Q}S_j \rightarrow \mathbb{R}$ je výplatní funkce hráče $i \in Q$.
            \end{itemize}

            Pro množinu hráčů $P \subseteq Q$ budeme $S_P$ označovat $\prod_{i \in P}S_i$. Doplněk $S_P$ bude označovat $S_{-P} = S_{Q\setminus P}$.

            Nechť $\alpha(K \subseteq Q) = \{x \in \mathbb{R}^{|Q|}\,|\, \exists\, a \in K:  x \leqslant a\}$.

            Mějme strategickou hru $\Gamma_S = (Q, (S_i)_{i\in Q}, (U_i)_{i\in Q})$. Tuto hru lze převést na ekvivalentní NTU--hru $\Gamma_N = (P, V)$, kde:
            \begin{itemize}
                \item $P = Q$,
                \item $V(K) = \begin{cases}
                    \alpha(\{(U_i(s))_{i\in P}\,|\,s \in S_P\}) \hspace*{2em} K = P\\
                    \emptyset \hspace*{11em} K = \emptyset \\[1.2em]
                    \begin{aligned}
                        \{a \in \mathbb{R}^{|P|}\,|\,&\exists s_K \in S_K \forall s_{-K} \in S_{-K}\\
                        &\forall i \in K: U_i(s_K, s_{-K}) \geq a_i\}\\
                        &\hspace*{7.3em}\textit{jinak}
                    \end{aligned}
                \end{cases}$
            \end{itemize}

        \subsubsection*{Kooperativní hry s vyjednáváním}
            \textit{\textbf{Kooperativní hra dvou hráčů $A$ a $B$ s vyjednáváním}} \cite{Hruby2022} je dvojice $\Gamma_B = (\Omega, c)$, kde $\Omega$ je množina všech dosažitelných výsledků a $c = (c_A, c_B)$ je výsledek, kte-rého by hráči dosáhli, pokud by se hra odehrála nekooperativně.  Existuje funkce $F$ modelující \textit{Nashovo vyjednávací řešení}. Je definována následně: $F(\Omega, c) = arg max_{u_A, u_B}\{(u_A - c_A)(u_B - c_B)\,|\, (u_A, u_B) \in \Omega \land u_A \geq c_A \land u_B \geq c_B\}$.

            Dvouhráčovou hru s vyjednáváním $\Gamma_B = (\Omega, c)$ s~funkci $F$ modelující Nashovo vyjednávací řešení lze převést na NTU--hru $\Gamma_N = (Q, V)$, kde:

            \begin{itemize}
                \item $Q = \{A, B\}$,
                \item $V(K) = \begin{cases}
                    (-\infty, c_A\rangle & K = \{A\}\\
                    (-\infty, c_B\rangle & K = \{B\}\\
                    F(\Omega, c) & K = \{A, B\}
                \end{cases}$
            \end{itemize}
        \subsubsection*{Veřejná volba}
            Převod veřejné volby na NTU--hru uvedeme na příkladu \cite[str. 120]{Game_Theory}. Mějme veřejnou volbu hráčů $1, 2, 3$ mezi alternativami $a_1, a_2$. Váhy jejich preferencí se různí a jsou určeny tabulkou:

            \begin{table}[!h]
                \centering
                \begin{tabular}{|c||c|c|}
                    \hline
                    & $a_1$ & $a_2$\\
                    \hline\hline
                    1 & 5 & 1 \\
                    \hline
                    2 & 2 & 3 \\
                    \hline
                    3 & 4 & 3 \\
                    \hline
                \end{tabular}
            \end{table}

            Nechť $\alpha(K \subseteq Q) = \{x \in \mathbb{R}^{|Q|}\,|\, \exists\, a \in K:  x \leqslant a\}$. Tento příklad veřejné volby odpovídá třihráčové NTU--hře $\Gamma = (Q, V)$, kde:
            \begin{itemize}
                \vspace*{-0.3em}
                \item $Q = \{1,2,3\}$
                \item $V(K) = \begin{cases}
                    (-\infty, 1\rangle & K = \{1\}\\
                    (-\infty, 2\rangle & K = \{2\}\\
                    (-\infty, 3\rangle & K = \{2\}\\
                    \alpha(\{(5, 2), (1, 3)\}) & K = \{1,2\}\\
                    \alpha(\{(5, 4), (1, 3)\}) & K = \{1,3\}\\
                    \alpha(\{(2, 4), (3, 3)\}) & K = \{2, 3\}\\
                    \alpha(\{(5, 2, 4), (1, 3, 3)\}) & K = \{1,2,3\}
                \end{cases}$
            \end{itemize}

        \subsubsection*{TU--hry}
            Jak již bylo dříve řečeno, TU--hry jsou pouhou podmnožinou NTU--her, tedy lze každou TU--hru převést na ekvivalentní NTU--hru.

            Ke každé TU--hře $\Gamma_T = (Q_T, v_T)$ existuje ekvivalentní NTU--hra $\Gamma_N = (Q_N, V_N)$, kde:

            \vspace*{-0.3em}
            \begin{itemize}
                \item $Q_N = Q_T$
                \item $V_N(K) = \begin{cases}
                            \emptyset & K = \emptyset\\
                            \{a \in \mathbb{R}^{|Q|}\,|\, \sum_{i\in K} a_i \leq v_T(K)\} &jinak
                        \end{cases}
                        $
            \end{itemize}

            Platí, že pokud je TU--hra superaditivní, pak je i~ji odpovídající NTU--hra superaditivní.

    \subsection{Převod NTU--her na strategické hry}
        V předchozí části jsme ukázali způsob převodu nekooperativní strategické hry v normální formě na ekvivalentní NTU--hru. V této podsekci je popisován opačný převod \cite{NTU2STR}, kterým je převod NTU--hry na strategickou.

        Mějme danou NTU--hru $\Gamma_N = (Q, V)$. Tuto hru lze převést na ekvivalentní strategickou hru (\textit{claim game}) $\Gamma_S = (Q, (S_i)_{i\in Q}, (U_i)_{i \in Q})$. Pro každého hráče $i \in Q$ je množina strategií $S_i = P_i \times \mathbb{R}$, kde $P_i = \{K \in 2^Q\,|\, i \in K\}$. Výplatní funkce $U_i$ je definována následovně: $U_i((K_1, t_1), \dots, (K_n, t_n)) = t_i$ pokud $(t_j)_{j\in K_i} \in V(K_i) \land \forall j \in K_i: K_i = K_j$, v opačném případě je $U_i((K_1, t_1), \dots,\\ (K_n, t_n)) = min\{t_i, v(i) := max\{\{a_i\,|\, a \in V(\{i\})\}\}\}$.

        Strategii $(K_i, t_i) \in P_i \times \mathbb{R}$ hráče $i \in Q$ leze chápat jako jaho přání na zformování koalice $K_i$ se ziskem $t_i$. Pokud takové přání mají všichni ostatní členové $j \in K_i$ s výplatami $(t_j)_{j\in K_i}$, pak hráči zformují koalici $K_i$ a~hráč $i \in Q$ získá $t_i$. V opačném případě dosáhne hráč $i$ při koalici $K_i$ na zisk $v(i)$, nebo $t_i$, pokud je $t_i < v(i)$.

\section{Jádro v NTU--hře}
    \label{sec:Core}
    Tato sekce uvádí definici a základní vlastnosti jádra NTU--her, které jsou zobecněním jádra TU--her. Jádro NTU--her je pochopitelně z důvodu nepřenositelnosti užitku obtížnější pro analýzu.

    \subsection{Pareto optimalita a Individuální racionalita}
        Nechť je $\Gamma = (Q, V)$ NTU--hra, $K \subseteq Q: K \neq \emptyset$ koalice a $x, y \in \mathbb{R}^{|Q|}$ dva výplatní vektory. Říkáme, že \textit{$y$ dominuje nad $x$ pro koalici $K$} pokud $y \in V(K)$ a $y \gg x$. Výplatní vektor \textit{$y$ dominuje nad $x$}, pokud existuje koalice $L \subseteq Q: L \neq \emptyset$, taková, že $y$ dominuje nad $x$ pro koalici $L$.

        Mějme NTU--hru $\Gamma = (Q, V)$, potom je \textit{\textbf{množina Pareto efektivních}} výplatních vektorů pro koalici $K \subseteq Q$ dána množinou $V(K)_e = \{a \in V(K)\,|\,\forall x \in V(K) \exists i \in Q: a_i \geq x_i\}$. Připomeňme, že množina $V(K)_e$ tvoří hranici množiny $V(K)$. Tato vlastnost vyplývá z axiomů A2 a A4.

        \textit{Maximální individuální zisk} hráče $i \in Q$ je maximum ze všech jeho zisků, kterých dosáhne, pokud se nezúčastní žádné koalice. Toto maximum budeme označovat $v^i$ a je definováno jako $v^i = max\{a_i\,|\, a \in V({i})\}$.

        Pro hru $\Gamma = (Q, V)$ je výplatní vektor $a$ \textit{Indivi-duálně racionální}, pokud $x_i \geq v^i$ pro všechny hráče $i \in Q$. \textit{\textbf{Množina Individuálně racionálních}} výplatních vektorů pro koalici $K \subseteq Q$ je pak dána množinou $V_{ir}(K) = \{a \in V(K)\,|\, a \textit{ je individuálně racionální}\}$.

    \subsection{Jádro hry}
        Jednou s hlavních důležitých otázek při analýze superaditivních her s~nepřenositelným užitkem je neprázdnost jádra. Pokud je jádro hry prázdné, pak nemá hra stabilní kooperativní řešení v podobě velké koalice.

        Nechť je $\Gamma = (Q, V)$ NTU--hra a $A \subseteq \mathbb{R}^{|Q|}$ podmnožina výplatních vektorů. \textit{\textbf{Jádro hry $\Gamma$ s ohledem na $A$}} je taková množina $C(Q, V, A) \subseteq A$, že pro každý výplatní vektor $a \in C(Q, V, A)$ platí, že není dominován žádným jiným vektorem z $A$.

        \textit{\textbf{Jádro hry}} $\Gamma = (Q, V)$ je množina výplatních vektorů $C(Q, V) \subseteq V(Q)$, pro které platí, že nejsou dominovány žádným jiným výplatním vektorem.

        Pro každou superaditivní NTU--hru $\Gamma = (Q, V)$ platí, že $C(Q, V) = C(Q, V, V_e(Q)) = C(Q, V, V_{ir}(Q)) = C(Q, V, V_e(Q) \cap C_{ir}(Q))$.

        Následující příklad popisuje jádro v třihráčové NTU--hře $\Gamma = (Q, V)$, kde:

        \begin{itemize}
            \vspace*{-0.3em}
            \item $Q = 1, 2, 3$
            \item $V(K) = \begin{cases}
                \{a \in \mathbb{R}^{|Q|}\,|\, a \leqslant (0.5, 0.5, 0)\} & \hspace*{-0.5em}K = Q\\
                \{a \in \mathbb{R}^{|Q|}\,|\, a_1 + a_2 \leq 1\} & \hspace*{-2em}K = \{1, 2\}\\
                \{a \in \mathbb{R}^{|Q|}\,|\, \forall i \in K: x_i \leq 0\} & \hspace*{-0.5em}K \subset Q
            \end{cases}$
        \end{itemize}

        Pro výše uvedený příklad třihráčové NTU--hry je jádro $C(Q, V) = \{(0.5, 0.5, 0)\}$.

\section{Shapleyho hodnota v NTU--hře}
    \label{sec:Shapley}
    V TU--hrách umožňuje Shapleyho hodnota \cite{Shapley1953} určit takové rozdělení zisků (výplatní vektor) koalice mezi její členy, že bude zisk každého hráče odrážel jeho přínos do koalice. Tento přístup zobecnil Shapley také pro NTU--hry \cite{Shapley1969}. V NTU--hrách je na základě Shapleyho NTU--hodnoty (známá také jako $\lambda$--přenosová NTU--hodnota) zvolen "spravedlivý" výplatní vektor z~množiny $V(Q)$.

    \subsection{Shapleyho NTU--hodnota}
        Mějme danou NTU--hru $(Q, V)$ a nechť $\Delta = \{\lambda \in \mathbb{R}^{|Q|}\,|\, \sum_{i\in Q}\lambda_i = 1\}$ je \textit{množina váhových vektorů}. Pro každý vektor $\lambda \in \Delta$ je definována mapovací funkce $v_\lambda: 2^Q \rightarrow \mathbb{R} \cup \{\infty\}$ následovně: $v_\lambda(K) = \{\lambda \cdot x\,|\, x \in V(K)\}$ pro $K \subseteq 2^Q \setminus \{\emptyset\}$ a $v_\lambda(\emptyset) = 0$. Váhový vektor $\lambda \in \Delta$ je \textit{proveditelný}, pokud pro každou koalici $K \subseteq Q$ platí $v_\lambda(K) \in \mathbb{R}$.

        Pro hru $\Gamma = (Q, V)$ je \textit{\textbf{Shapleyho NTU--hodnota}} vektor $a_{SH} \in V(Q)$, pokud existuje proveditelný váho-vý vektor $\lambda \in \Delta$ takový, že $\phi(v_\lambda) = \lambda a_{SH}$. Množina všech Shapleyho NTU--hodnot je $SH(V)$ (hra může mít více Shapleyho NTU--hodnot).

        Tedy spravedlivým výplatním vektorem pro koalici $Q$ je takový vektor $a$, jehož vektorový součin s provedi-telným váhovým vektorem $\lambda$ se rovná Shapleyho hodnotě pro TU--hru, specifikovanou charakteristickou funkcí $v_\lambda$.

        Mějme kooperativní vyjednávací hru $\Gamma_B = (\Omega, c)$ a ji odpovídající NTU--hru $\Gamma_N = (Q, V)$, pak platí, že $HS(V) = \{F(\Gamma, c)\}$.

        Nechť je $\Gamma_T = (Q, v_T)$ TU--hra a $\Gamma_N = (Q, V_N)$ ji odpovídající NTU--hra, pak platí, že $HS(V_N) = \{\phi(v_T)\}$.

        Pokud má NTU--hra $\Gamma = (Q, V)$ následující vlastnosti: pro každou koalici $L \in 2^Q \setminus \{\emptyset\}$ existuje konvexní kompaktní množina $C(L) \in \mathbb{R}^{|Q|}$ a konvexní kužel $K(L) \in \mathbb{R}^{|Q|}$ takový, že:
        \begin{itemize}
            \item $V(L) = C(L) + K(L)$,
            \item $K(Q) \supseteq K(L) \times \{0^{Q\setminus L}\}$,
            \item $\forall x, y \in \partial V(Q): x \geqslant y \implies x = y$,
        \end{itemize}
        pak pro hru $\Gamma$ existuje Shapleyho NTU--hodnota.

    \subsection{Axiomy Shapleyho NTU--hodnoty}
        Mějme NTU--hry $\Gamma_U = (Q, U), \Gamma_V = (Q, V)$ a $\Gamma_W = (Q, W)$. Pro Shapleyho NTU--hodnoty platí následující axiomy.

        \subsubsection*{A8: Pareto efektivita}
            Platí $SH(V) \subseteq \partial V$. Tedy každý výplatní vektor $a \in SH(V)$ splňuje Pareto efektivitu pro všechny $K \subseteq Q$ podle A4.

        \subsubsection*{A9: Měřítková kovariance}
            Podle měřítkové kovariance (\textit{scale covariance}) platí: $SH(\lambda V) = \lambda SH(V)$ pro všechny vektory $\lambda \in \mathbb{R}^{|Q|}_{+}$. Jinak řečeno: výsledek hry s výplatními vektory vynásobenými libovolným kladným vektorem $\lambda$ jsou totožné s výsledky původní hry vynásobenými tímto vektorem.

        \subsubsection*{A10: Podmíněná aditivita}
            Pokud $U = V + W$, potom platí $SH(U) \supseteq (SH(V) + SH(W)) \cap \partial U$. Tedy nechť jsou $x \in SH(V)$ a $y \in SH(V)$ řešením her $\Gamma_V$ a $\Gamma_W$, potom pokud je $z = x + y$ pareto efektivní pro $U$, tak je $z$ řešením hry $\Gamma_U$ ($z \in SH(U)$).
        \subsubsection*{A11: Nezávislost na irelevantních alternativách}
            Pokud $V(Q) \subseteq W(Q)$ a $V(K) = W(K)$ pro všechny $K \varsubsetneqq Q$, potom $HS(V) \supseteq HS(V) \cap V(Q)$. Jinak řečeno: pokud je $a \in HS(W)$ řešením ve hře $\Gamma_W$ a hra $\Gamma_W$ je podhrou $\Gamma_V$, pak je $a$ také řešením pro $\Gamma_V$
        \subsubsection*{A12: Unanimity--hra}
            Pro každou koalici $K \in 2^Q \setminus \{\emptyset\}$ a každé reálné číslo $c \in \mathbb{R}$ je definována NTU unanimity--hra na koalic $K$ jako $\Gamma^K_{UN} = (Q_{UN}, V_{UN})$, kde:
            \begin{itemize}
                \item $Q_{UN} = Q$
                \item $V_{UN}(L) = \begin{cases}
                    \{a \in \mathbb{R}^{|Q_{UN}|}\,|\,a(L) \leq 1\} & \hspace*{-0.8em},Q \supseteq L \supseteq K\\
                    \{a \in \mathbb{R}^{|Q_{UN}|}\,|\,a(L) \leq 0\} & \hspace*{-0.8em},Q \neq L \nsubseteq  K
                \end{cases}$
            \end{itemize}
            Nyní mějme vektor $z \in \mathbb{R}^{|Q|}$, kde $z_i = c/|K|$, pokud $i \in K$, jinak $z_i = 0$. Potom platí, že $SH(\Gamma^K_{UN}) = \{z\}$.

\section{Závěr}
    V této práci byla popsána struktura kooperativní hry s nepřenositelným užitkem, která je tvořena množinou hráčů a charakteristickou funkci, která pro každou koalici poskytuje množinu možných výplatních vektorů.

    Hra s přenositelným užitkem je speciální podtřídou NTU--her, proto ji lze na NTU--hru převést. Tento převod společně s převody ostatních her na NTU--hru byl uveden v sekci \ref{sec:NTU}.

    Podstatnou vlastností NUT--hry je její jádro, které obsahuje pareto efektivní a individuálně racionální výplatní vektory. Právě analýzou jádra lze určit, zda zadaná hra má kooperativní řešení v podobě velké koalice (jádro je neprázdné).

    Posledním popisovaným konceptem pro analýzu NTU--her byla Shapleyho NTU--hodnota, která určuje "spravedlivé" výplatní vektory pro koalici, s ohledem na hráčův přínos koalici (respektive ztrátu způsobenou při vystoupení hráče z koalice).

\else
    \input{xsedym02-NTU-text-en.tex}
\fi

%--------------------------------------------------------
%--------------------------------------------------------
%--------------------------------------------------------
%	LITERATURA
%--------------------------------------------------------
%--------------------------------------------------------
\phantomsection
\ifczech
    \bibliographystyle{bib-styles/czplain}
\fi
\ifslovak
    \bibliographystyle{bib-styles/skplain}
\fi
\ifenglish
    \bibliographystyle{bib-styles/enplain}
\fi

\bibliography{xsedym02-NTU-bib}

%--------------------------------------------------------
%--------------------------------------------------------
%--------------------------------------------------------
\end{document}