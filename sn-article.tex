%Version 3 December 2023
% See section 11 of the User Manual for version history
%
%%%%%%%%%%%%%%%%%%%%%%%%%%%%%%%%%%%%%%%%%%%%%%%%%%%%%%%%%%%%%%%%%%%%%%
%%                                                                 %%
%% Please do not use \input{...} to include other tex files.       %%
%% Submit your LaTeX manuscript as one .tex document.              %%
%%                                                                 %%
%% All additional figures and files should be attached             %%
%% separately and not embedded in the \TeX\ document itself.       %%
%%                                                                 %%
%%%%%%%%%%%%%%%%%%%%%%%%%%%%%%%%%%%%%%%%%%%%%%%%%%%%%%%%%%%%%%%%%%%%%

%%\documentclass[referee,sn-basic]{sn-jnl}% referee option is meant for double line spacing

%%=======================================================%%
%% to print line numbers in the margin use lineno option %%
%%=======================================================%%

%%\documentclass[lineno,sn-basic]{sn-jnl}% Basic Springer Nature Reference Style/Chemistry Reference Style

%%======================================================%%
%% to compile with pdflatex/xelatex use pdflatex option %%
%%======================================================%%

%%\documentclass[pdflatex,sn-basic]{sn-jnl}% Basic Springer Nature Reference Style/Chemistry Reference Style


%%Note: the following reference styles support Namedate and Numbered referencing. By default the style follows the most common style. To switch between the options you can add or remove “Numbered” in the optional parenthesis.
%%The option is available for: sn-basic.bst, sn-vancouver.bst, sn-chicago.bst%

%%\documentclass[pdflatex,sn-nature]{sn-jnl}% Style for submissions to Nature Portfolio journals
% \documentclass[pdflatex,sn-basic]{sn-jnl}% Basic Springer Nature Reference Style/Chemistry Reference Style
\documentclass[pdflatex,sn-mathphys-num]{sn-jnl}% Math and Physical Sciences Numbered Reference Style
% \documentclass[pdflatex,sn-mathphys-ay]{sn-jnl}% Math and Physical Sciences Author Year Reference Style
%%\documentclass[pdflatex,sn-aps]{sn-jnl}% American Physical Society (APS) Reference Style
%%\documentclass[pdflatex,sn-vancouver,Numbered]{sn-jnl}% Vancouver Reference Style
% \documentclass[pdflatex,sn-apa]{sn-jnl}% APA Reference Style
%%\documentclass[pdflatex,sn-chicago]{sn-jnl}% Chicago-based Humanities Reference Style

%%%% Standard Packages
%%<additional latex packages if required can be included here>

\usepackage{graphicx}%
\usepackage{multirow}%
\usepackage{amsmath,amssymb,amsfonts}%
\usepackage{amsthm}%
\usepackage{mathrsfs}%
\usepackage[title]{appendix}%
\usepackage{xcolor}%
\usepackage{textcomp}%
\usepackage{manyfoot}%
\usepackage{booktabs}%
\usepackage{algorithm}%
\usepackage{algorithmicx}%
\usepackage{algpseudocode}%
\usepackage{listings}%
%%%%

%%%%%=============================================================================%%%%
%%%%  Remarks: This template is provided to aid authors with the preparation
%%%%  of original research articles intended for submission to journals published
%%%%  by Springer Nature. The guidance has been prepared in partnership with
%%%%  production teams to conform to Springer Nature technical requirements.
%%%%  Editorial and presentation requirements differ among journal portfolios and
%%%%  research disciplines. You may find sections in this template are irrelevant
%%%%  to your work and are empowered to omit any such section if allowed by the
%%%%  journal you intend to submit to. The submission guidelines and policies
%%%%  of the journal take precedence. A detailed User Manual is available in the
%%%%  template package for technical guidance.
%%%%%=============================================================================%%%%

%% as per the requirement new theorem styles can be included as shown below
\theoremstyle{thmstyleone}%
\newtheorem{theorem}{Theorem}%  meant for continuous numbers
%%\newtheorem{theorem}{Theorem}[section]% meant for sectionwise numbers
%% optional argument [theorem] produces theorem numbering sequence instead of independent numbers for Proposition
\newtheorem{proposition}[theorem]{Proposition}%
%%\newtheorem{proposition}{Proposition}% to get separate numbers for theorem and proposition etc.

\theoremstyle{thmstyletwo}%
\newtheorem{example}{Example}%
\newtheorem{remark}{Remark}%

\theoremstyle{thmstylethree}%
\newtheorem{definition}{Definition}%

\raggedbottom
% \unnumbered% uncomment this for unnumbered level heads

\begin{document}

\title[Article Title]{Mona Reimplemented: WS1S Logic with Mata}

%%=============================================================%%
%% GivenName	-> \fnm{Joergen W.}
%% Particle	-> \spfx{van der} -> surname prefix
%% FamilyName	-> \sur{Ploeg}
%% Suffix	-> \sfx{IV}
%% \author*[1,2]{\fnm{Joergen W.} \spfx{van der} \sur{Ploeg}
%%  \sfx{IV}}\email{iauthor@gmail.com}
%%=============================================================%%

\author{\fnm{Michal} \sur{Šedý}}\email{xsedym02@stud.fit.vutbr.cz}

%%==================================%%
%%     Unstructured abstract        %%
%%==================================%%

\abstract{This paper focuses on the reimplementation of the decision procedure for WS1S logic, a second-order logic that can be decided using finite automata. The well known tool for WS1S logic decision, Mona, employs automata with transitions represented through binary decision diagrams (BDDs). Due to the integration of BDDs in automata operations, tasks like reversal cannot be executed in the conventional manner of reverting individual edges. Instead, the reversal of each BDD must be computed, potentially resulting in an exponential blowup. Motivated by these limitations, Pavel Bednar reimplemented Mona using a pure automata approach with the Mata library. This work optimizes the automata methodology, resulting in a significant speedup, up to ten times faster, in WS1S decision compared to Bednar's original reimplementation.}


\keywords{Finite Automata, Binary Decision Diagrams, WS1S, MONA, MATA}


\maketitle


%%==================================%%
%%          INTRODUCTION            %%
%%==================================%%

\section{Introduction}
    The most well known decision procedures are SAT and SMT \cite{SAT_SMT}, that are widely used in various applications such as verification (e.g., predicate abstraction), test generation, hardware synthesis, minimization, artificial intelligence, etc. The SAT (satisfiability) problem is a decision problem which asks whether a given propositional formula is satisfiable. The SMT (satisfiability modulo theories) problem extends the SAT problem to satisfiability of first-order formulae with equality and atoms from various first-order theories. There are various high-order decision procedures such as WS1S, WS2S, WSkS, S1S, etc.

    This work focuses on the WS1S, weak monadic second-order theory of first successor. The word weak stands for finite sets, monadic indicates unary relations, second-order allows usage of quantifier over the relations, and first successors means that there is only one successor (e.g. the structure is linear). WS1S \cite{WS1S} has an extremely simple syntax and semantics: it is variation of predicate logic with first-order variable that denote natural numbers and second-order variables that denote finite sets of natural numbers, it has a single function symbol, which denotes the successor function and has usual comparison operators such as $\leq$, $=$, $\in$ and $\supseteq$. Richard Büchi presented approach how to decide WS1S using finite automata in \cite{Buchi} The main idea is to recursively transforms each subformula of the main WS1S formula into deterministic finite automata (DFA) representing feasible interpretations and simulate boolean operations via the operation over automata.

    The most used tool for deciding WS1S and WS2S is Mona\footnote{accessible at \url{https://www.brics.dk/mona/index.html}} which uses Büchi's recursion approach for the construction of finite automaton with binary decision diagram (BDD) for the representation of all automaton's transitions. The usage of BDD makes operations over automata faster but in the cost of making come operations such as reversion expensive (potential exponential blowup). Despite this limitation Mona is used in various field of program verification such as:  verification of programs with complex dynamic data structures \cite{DDS1, DDS2}, string analysis \cite{string_analysis}, parametrized systems \cite{parametrized_systems} distributed systems \cite{distributed_systems}, automatic synthesis \cite{automatic_synthesis} hardware verification \cite{hardware_verification} and many others.

    The previously mentioned problem with hard to compute automata operations when using BDDs motivated Bc. Pavel Bednář's master's thesis. The new pure automata based approach v



% \begin{appendices}

% \section{Section title of first appendix}\label{secA1}

% An appendix contains supplementary information that is not an essential part of the text itself but which may be helpful in providing a more comprehensive understanding of the research problem or it is information that is too cumbersome to be included in the body of the paper.

% %%=============================================%%
% %% For submissions to Nature Portfolio Journals %%
% %% please use the heading ``Extended Data''.   %%
% %%=============================================%%

% %%=============================================================%%
% %% Sample for another appendix section			       %%
% %%=============================================================%%

% %% \section{Example of another appendix section}\label{secA2}%
% %% Appendices may be used for helpful, supporting or essential material that would otherwise
% %% clutter, break up or be distracting to the text. Appendices can consist of sections, figures,
% %% tables and equations etc.

% \end{appendices}

%%===========================================================================================%%
%% If you are submitting to one of the Nature Portfolio journals, using the eJP submission   %%
%% system, please include the references within the manuscript file itself. You may do this  %%
%% by copying the reference list from your .bbl file, paste it into the main manuscript .tex %%
%% file, and delete the associated \verb+\bibliography+ commands.                            %%
%%===========================================================================================%%

\bibliography{sn-bibliography}% common bib file
%% if required, the content of .bbl file can be included here once bbl is generated
%%\input sn-article.bbl


\end{document}
